\documentclass[11pt,a4paper]{article}
\usepackage{graphics}
\usepackage{alltt}
\usepackage[portuges]{babel}
\usepackage[utf8x]{inputenc}
\usepackage{color}
\usepackage{fancyhdr}
\usepackage{listings}
\usepackage{multicol}
\usepackage{babel}
\usepackage{hyperref}
\usepackage{t1enc}
%\usepackage{aeguill}
\usepackage{varioref}
\usepackage{fancyvrb}
\usepackage{color}
\usepackage{cite}
\usepackage{float}
\usepackage{listings}
\usepackage{harvard}
\usepackage{geometry}
\usepackage{setspace}
\usepackage{thumbpdf}
\usepackage{fancyhdr}
\usepackage{lastpage}
\usepackage{alltt}
\usepackage{graphicx}
\usepackage{verbatim}
\usepackage{color}
\usepackage{url}
\usepackage{epsf}
\usepackage{listings}
\usepackage[refpages]{gloss}
\usepackage{array}
\usepackage{longtable}
\usepackage{multirow}
\usepackage{amsmath, amsthm, amssymb}
\usepackage{slashbox}
\usepackage{rotating}

\setlength{\textwidth}{16.5cm}
\setlength{\textheight}{24cm}
\setlength{\parindent}{1em}
\setlength{\parskip}{0pt plus 1pt}
\setlength{\oddsidemargin}{0cm}
\setlength{\evensidemargin}{0cm}
\setlength{\topmargin}{-1.1cm}
\setlength{\headsep}{20pt}
\setlength{\columnsep}{1.5pc}
\setlength\columnseprule{.4pt}
\setlength\premulticols{6\baselineskip}
\pagestyle{fancy}

\definecolor{castanho_ulisses}{rgb}{0.71,0.33,0.14}
\definecolor{gray_ulisses}{gray}{0.55}
\definecolor{green_ulises}{rgb}{0.2,0.75,0}

\lstdefinelanguage{C_ulisses}{
	basicstyle=\ttfamily\scriptsize,
    sensitive=true,
	morecomment=[l][\color{gray_ulisses}\scriptsize]{//},
	morecomment=[s][\color{gray_ulisses}\scriptsize]{/*}{*/},
	morestring=[b]",
	morestring=[b]',
	stringstyle=\color{red},
	showstringspaces=false,
	numbers=left,
	numberstyle=\tiny,
	numberblanklines=true,
	showspaces=false,
	showtabs=false,
	xleftmargin=-10pt,
	xrightmargin=-20pt,
	emph=
	{[1]
		return,printf,fprintf,if,else,scanf,sscanf,sprintf,malloc,
		calloc,realloc,fgets,fputs,puts,system,strcmp,strstr,
		strchr,exit,for,while,fclose,fopen,atoi,sizeof,sin,cos
	},
	emphstyle={[1]\color{blue}\textbf},
	emph={[2]char,float,double,void,NULL,int,bool},
	emphstyle={[2]\color{green_ulises}\textbf},
	emph={[3]typedef, struct},
	emphstyle={[3]\color{castanho_ulisses}\textbf}
}

\lstdefinelanguage{txt}
{
       basicstyle=\ttfamily\scriptsize,
       showstringspaces=false,
       numbers=left,
       numberstyle=\tiny,
       numberblanklines=true,
       showspaces=false,
       showtabs=false
}
\lstnewenvironment{code_txt}{\lstset{language=txt}}{}
\lstnewenvironment{code_c}{\lstset{language=C_ulisses}}{}

\lstnewenvironment{code_lex}
{\textbf{Código Lex} \hspace{1cm} \hrulefill \lstset{language=txt}}
{\hrule\smallskip}

\title{\sf  Engenharia de Linguagens \\ Engenharia Gramatical \\
\begin{tabular}{c}
    \includegraphics[width=.1\textwidth]{stuff/uminho.jpg}
    \includegraphics[width=.07\textwidth]{stuff/informatica.jpg}\\
    {\small Universidade do Minho}, {\small MEI}\\
    {\small Ano lectivo 2010/2011}\\
    {\small Trabalho Prático}\\
\end{tabular}
}
\author{
    {\small José Pedro Silva - pg17628} \and
    {\small Mário Ulisses Costa - pg15817} \and
    {\small Pedro Faria - pg17684}}
\date{{\small \today}}

\begin{document}
\maketitle

\begin{abstract}
Este relatório diz respeito a um trabalho de Engenharia de Linguagens que se divia em duas partes/problemas:
\begin{itemize}
 \item Transformação da gramática da linguagem \texttt{LogoLISS} feita em \texttt{Yacc} para \texttt{AntLR} 
 \item Parser da linguagem e posterior execução.
\end{itemize}

\end{abstract}

\tableofcontents

\newpage
\section{Problema - Parte 1}
A ideia deste trabalho seria implementar um programa que transformasse gramáticas em \texttt{Yacc} para \texttt{AntLR}. Para isso, escolheu-se fazer uma gramática em 
\texttt{Yapp} ( um módulo de \texttt{Perl} próprio para implementação de gramáticas ) para tratar do ficheiro \texttt{.y} e um pequeno parser em \texttt{Perl} para retirar os 
\emph{tokens} do fichero \texttt{.l}. \\

Para testar, usou-se um trabalho realizado para a cadeira de Processamento de Linguagens lecionada no ano passado. 

\section{Solução}

As subsecções que se seguem referem-se à resolução para cada ficheiro pretendido, cada um contendo informação acerca da estrutura, das dificuldades e 
do estado da implementação.

\subsection{Lex}
Neste documento, estão contidas os \emph{tokens} da linguagem, que serão necessários para a gramática em \texttt{AntLR}. Assim, decidiu-se criar um \texttt{script} em \texttt{Perl}
que fizesse \emph{parse} desse ficheiro, de modo a devolver um ficheiro contendo os \emph{tokens} definidos no modo \texttt{AntLR} ou respectivas produções para não terminais.

Encarou-se um documento em \texttt{Lex} como estando estruturado da seguinte forma:\\

\begin{itemize}
 \item Um cabeçalho, onde se encontram definições, \emph{flags} e outras informções relevantes.
 \item O corpo do documento, divido, ou não, em estados. Cada estado contendo uma expressão regular (\textbf{er}) para fazer correspondência com o texto fornecido, as expressões
 ordenadas de forma crescente consoante o tamanho do seu domínio.
 \item Um terceiro espaço reservado, essencialmente, para funções.
\end{itemize}

Cada espaço encontra-se separado por \texttt{\%\%}, como se pode observar no seguinte pequeno exemplo.\\

\begin{code_txt}
IDENT           [a-zA-Z][a-zA-Z0-9]*

\%\%

{IDENT}                 {yylval.vals=(char*)strdup(yytext);return(IDENT);}
" "|\n|\t               {;}

\%\%

int yywrap()
{ return(1); }

\end{code_txt}

\paragraph{Dificuldades} foram encontrados para o \emph{parsing} deste ficheiro. Principalmente no facto de no ficheiro \texttt{.l} com que se trabalhou não ter definido 
nenhum \emph{token} que fosse simplesmente uma palavra, sendo todos definidos como expressões regulares. Isso significa que não se poderia definir os \emph{tokens} em \texttt{AntLR}
da forma mais simples, que seria:\\

\begin{code_txt}
tokens{
	Simples = 'simples';
}
\end{code_txt}
Esse modo não permite definir os \emph{tokens} como expressões regulares, sendo a sua definição como um símbolo não terminal:\\

\begin{code_txt}
Simples	:	('S'|'s')('I'|'i')('M'|'m')('P'|'p')('L'|'l')('E'|'e')('S'|'s')
	;
\end{code_txt}

\paragraph{Outras dificuldades} foram as transformações das expressões regulares. A sua definição em \texttt{Lex} difere muito da sua definição em \texttt{AntLR}

\subsubsection{Resolução}
A resolução deste primeiro documento, trata de guardar todas as definições de expressões regulares e retorno de \emph{tokens} para duas estruturas \emph{hash} diferentes.
Como foi explicado anteriormente, o \emph{parser} funciona de maneira diferente consoante o estado em que se encontra.\\

Antes da primeira ocorrência do primeiro separador (o \texttt{\%\%}), guarda-se todas as definições para a \emph{hash} \texttt{definitions}. De seguida apresenta-se exemplos de 
como o texto está no ficheiro \texttt{Lex} e o pedaço de código que guarda essa informação.\\

\begin{code_txt}
PROGRAM         [Pp][Rr][Oo][Gg][Rr][Aa][Mm]
DECLARATIONS    [Dd][Ee][Cc][Ll][Aa][Rr][Aa][Tt][Ii][Oo][Nn][Ss]
STATEMENTS      [Ss][Tt][Aa][Tt][Ee][Mm][Ee][Nn][Tt][Ss]
\%\%
\end{code_txt}

A variável \texttt{\$flag} representa o estado em que se encontra.\\

\begin{code_c}
        if(\$flag == 0){
        \#       print "Definition State\n";
                m/(\w+)\s+(.+)/g;
                \$definitions{\$1} = \$2;

        }
\end{code_c}

Finalizando esta primeira fase, seguia-se a procura do retorno de \emph{tokens}, que se guardariam na \emph{hash} \texttt{hashtokens}. Como geralmente se encontram as definições
em vez das expressões regulares respectivas, procurou-se de imediato a expressão regular correspondente, visto que a sua definição de nada iria valer no processo final de transformação.
O código respectivo para esta fase:\\

\begin{code_c}
        elsif(\$flag == 1){
        \#       print "Token state\n"
                m/([^\t\n]+)\s+(\S+)/g;
                my \$aux = \$1;
                my \$aux2 = \$2; 

                if(\$aux =~ /^{[A-Za-z]+}\$/){
                        \$aux =~ s/^\{//g;
                        \$aux =~ s/\}\$//g;
                        \$hashtokens{\$aux} = \$definitions{\$aux};
                }
                print "RE: \$aux -> SEMANTICA: \$aux2 \n";
        }
\end{code_c}

O passo final de transformar as ers de \texttt{Lex} para a syntax \texttt{AntLR} ainda não foi concluído.

\subsection{Gramática}
O problema de tentar traduzir gramáticas \textbf{LALR} para \textbf{LL} não é recente, mas também não é um problema trivial. Tendo isto em mente e motivados pela proposta
para este trabalho que nos foi lançada, decidimos tentarmos fazer o nosso próprio conversor.\\
Começamos por investigar que tipo de regras seriam interessantes avaliar e verificar para este problema. Assim, conseguimos chegar com 5 regras, que nos parecem ser a chave
para o problema, eventualmente existem mais regras, mas estas pareceram-nos ser as mais importantes. E foram estas que mais tempo dedicamos a compreender e analizar.

\subsubsection{Regras}
\begin{figure}[H]
\begin{center}
\begin{tabular}{|l|l|c|}\hline
\textbf{LALR (BNF)} & \textbf{LL (EBNF)} & Nome da Regra\\\hline
\begin{tabular}{ccl}
vars & : & var\\
    & | & vars var
\end{tabular}
&
\begin{tabular}{ccl}
vars & : & (var)$+$
\end{tabular}
&
removeLeftRecursionPlus\\\hline
\begin{tabular}{ccl}
vars & : & \\
    & | & vars var
\end{tabular}
&
\begin{tabular}{ccl}
vars & : & (var)$*$
\end{tabular}
&
removeLeftRecursionTimes\\\hline
\begin{tabular}{ccl}
a & : & A b | A\\
b & : & B a | B
\end{tabular}
&
\begin{tabular}{ccl}
a & : & A (B a | B) | A\\
b & : & B a | B
\end{tabular}
&
removeCircularRecursion\\\hline
\begin{tabular}{ccl}
a & : & A b | A c
\end{tabular}
&
\begin{tabular}{ccl}
a & : & A ( b | c )
\end{tabular}
&
leftFactoring1\\\hline
\begin{tabular}{ccl}
a & : & b | c\\
b & : & A d\\
c & : & A e
\end{tabular}
&
\begin{tabular}{ccl}
a & : & A ( b | c )\\
b & : & d\\
c & : & e
\end{tabular}
&
leftFactoring2\\\hline
\end{tabular}
\caption{Regras de conversão de \textbf{LALR} para \textbf{LL}}
\end{center}
\end{figure}

Como podemos ver todas estas regras nos ajudam a fazer uma conversão automática entre estes dois tipos de gramáticas. Um pequena extensão poderia ser feita a ambas as regras
de remoção da recursividade à esquerda:

\begin{figure}[H]
\begin{center}
\begin{tabular}{|l|l|c|}\hline
\textbf{LALR (BNF)} & \textbf{LL (EBNF)} & Nome da Regra\\\hline
\begin{tabular}{ccl}
vars & : & var\\
    & | & vars a b c d e \ldots
\end{tabular}
&
\begin{tabular}{ccl}
vars & : & var (a b c d e \ldots)+
\end{tabular}
&
removeLeftRecursionPlusExt\\\hline
\begin{tabular}{ccl}
vars & : & \\
    & | & vars a b c d e \ldots
\end{tabular}
&
\begin{tabular}{ccl}
vars & : &  (a b c d e \ldots)*
\end{tabular}
&
removeLeftRecursionTimesExt\\\hline
\end{tabular}
\caption{Regras para remoção da recursividade à esquerda extendidas}
\end{center}
\end{figure}

\subsubsection{Implementação}
Para a implementação desta parte do conversor decidimos usar o \textbf{yapp}, porque já temos algum avontade com a linguagem \textbf{Perl} e por já termos
mexido neste modulo para outras cadeiras. \\
A gramática que implementamos foi a seguinte:

\begin{code_txt}
productions : production             
            | productions production 
            ;

production  : nonTerminal ':' derivs ';' 
            ;

derivs      : nts            
            | derivs '|' nts 
            | '|' nts        
            | '|'            
            ;
nts         : nt     
            | nts nt 
            ;
nt          : terminal    
            | nonTerminal 
            | sep         
            ;
terminal    : STRING_TERMINAL     
            ;
nonTerminal : STRING_NON_TERMINAL
            ;
sep         : SEPARATOR 
            ;
\end{code_txt}

Esta foi a gramática que criamos para fazer parsing de uma cramática em \textbf{Yacc}.
A gramática acima descrita com acções semânticas escritas em \textbf{Perl} ficaria da seguinte forma:

\begin{code_txt}
productions : production             { return $_[1]; }
            | productions production {
                                       push @{$_[1]},@{$_[2]};
                                       return $_[1];
                                     }

            ;

production  : nonTerminal ':' derivs ';' { return [ { $_[1] => $_[3] } ]; }
            ;

derivs      : nts            { return [$_[1]]; }
            | derivs '|' nts {
                               push @{$_[1]},$_[3];
                               return $_[1];
			                 }
            | '|' nts        { return [ [ { 'epsilon' => 'epsilon' } ],$_[2]]; }
            | '|'            { return [ { 'epsilon' => 'epsilon' } ]; }
            ;
nts         : nt     { return $_[1]; }
            | nts nt {
                       push @{$_[1]},@{$_[2]};
					   return $_[1];
                     }
            ;
nt          : terminal    {
                            return [ { $_[1] => 'terminal' } ];
                          }
            | nonTerminal {
                            return [ { $_[1] => 'nonTerminal' } ];
                          }
            | sep         {
                            return [ { $_[1] => 'sep' } ];
                          }
            ;
terminal    : STRING_TERMINAL     {
                                    return $_[1];
                                  }
            ;
nonTerminal : STRING_NON_TERMINAL {
                                    return $_[1];
                                  }
            ;
sep         : SEPARATOR           {
                                    return $_[1];
                                  }
            ;
\end{code_txt}

Depois de executar este código conseguimos obter de forma estruturada a gramática do \textbf{Yacc}.\\
Esta estrutura foi pensada para comportar uma gramática onde fosse fácil fazer cálculos sobre ela, como por exemplo a aplicação das regras atrás mencionadas.\\

Esta estrutura é a seguinte (utilizando os tipos do \textbf{Haskell}):
$$type~Grammar = [Production]$$

$$type~Production = HashTable~ProductionName~[Derivation]$$

$$type~Derivation = [HashTable~NTS~Type]$$

$$type~Type = String$$

$$type~NT = String$$
Onde $Type$ pode ser do tipo $nonTerminal$, $terminal$ ou $separator$ e $NT$ representa o nome do não terminal, terminal ou separador.
\newpage
\section{Probelma - Parte 2}
A segunda parte do problema passa por usar a gramática gerada na primeira parte do problema.\\

Não tendo problemas concretos com alíneas para solucionar, o trabalho realizado passou por colher as informações no bloco \texttt{Declarations} e processar a informação do bloco 
\texttt{Statements}

\subsection{Classes Java}
Antes de proceder à explicação, é importante explicar a criação das seguintes classes em \texttt{Java}:
\begin{itemize}
 \item \textbf{Variavel} - Classe que guarda informação sobre as variáveis encontradas: o nome, o tipo, o valor e a lista de elementos (no caso do tipo ser \texttt{Array}, caso contrário 
\emph{null})
 \item \textbf{Tipo} - Não sendo uma classe, é uma enumeração dos vários tipos existentes na linguagem: \texttt{Integer}, \texttt{Boolean} e \texttt{Array}.
 \item \textbf{ListaVar} - Classe que guarda uma \emph{hash} de variáveis, em que a sua chave é o nome da variável. \textbf{Apesar de o nome não o sugerir}, esta classe serviu também para
dar um ``estado'' ao que se sugeriu ser uma tartaruga (Mr. Turtle).
 \item \textbf{Position} - Classe que apenas contém duas variáveis para se referir à posição da tartaruga.
\end{itemize}

\subsection{Bloco \texttt{Declarations}}
Nesta fase, faz-se uma recolha de todas as variáveis declaradas e respectivas informações.\\

Os dois seguintes pedaços de código são os mais relevantes. O primeiro diz respeito á informção que chega à produção \texttt{variable\_declaration : vars '->' type}

\begin{code_txt}
 type    
returns[Tipo t_out, int s_out]
	:    INTEGER				{$t_out = Tipo.Integer; $s_out = 0;}
    	|    BOOLEAN				{$t_out = Tipo.Boolean; $s_out = 0;}
    	|    ARRAY    SIZE    NUM		
{$t_out = Tipo.Array; $s_out = Integer.parseInt($NUM.text);}
    	;
        
vars
returns[ArrayList<Variavel> lista_out]
@init{ArrayList<Variavel> lista = new ArrayList<Variavel>();}    
	:    var{lista.add($var.v_out);}    
(','    vs=vars {for(Variavel va : $vs.lista_out) lista.add(va);})? {$lista_out = lista;}
    	;

var
returns[Variavel v_out]
@init{Variavel v = new Variavel();}
	:    IDENT{v.nome = $IDENT.text;}    vv=value_var 
{v.valor = $vv.value_out; v.lista = $vv.lista_out; $v_out = v;}
    	;
\end{code_txt}

Como na produção mencionada anteriormente se pode verificar, o tipo das variáveis vem depois da declaração das mesmas. Isto levanta problemas que podem ser soluccionados de duas maneiras.
Uma seria modificar a AST de modo a manipulá-la e conseguir chegar a informação do tipo antes. A escolhida, foi apanhar as variáveis e só então declarar o seu tipo. Esta solução, exigiu 
alguma agilidade na forma de adicionar correctamente os tipos. Assim, criou-se uma lista provisória de variáveis que passaria por tratamento antes de ser devolvidada a uma lista 
principal. O segundo pedaço de código diz respeito a esse tratamento, e, realça-se também o caso de especial de as variáveis serem \texttt{Arrays}.

\begin{code_txt}
 variable_declaration
returns[ArrayList<Variavel> lista_out]
    	:    vars{$lista_out = $vars.lista_out;}    '->'    tt=type
{	Tipo x = $tt.t_out;
	int tamanho = $tt.s_out;
	//System.out.println("Tamanho:" + $lista_out.size());
	for(Variavel va : $lista_out){
		va.tipo = x; 
		if(va.tipo == Tipo.Array){ 
			va.valor = tamanho;
		}
	} 
	if(x == Tipo.Array){
		for(Variavel var : $lista_out){
			int tam = var.lista.size();
			while(tam < var.valor){
				var.lista.add(0);
				tam++;
			}
		}
	}
}  ';'
    	;
\end{code_txt}

Quando finaliza este bloco, uma instância da classe \texttt{ListaVar} é criada, tendo como argumento, uma lista de variáveis temporária. Essa instância será depois um atributo herdado 
no bloco \texttt{Statements}, como se pode ver:

\begin{code_txt}
body
@init{ListaVar list;}
    	:    DECLARATIONS        dec=declarations		
    	     STATEMENTS {//for(Variavel v : $dec.lista_out) System.out.println(v.toString());
    	     			list = new ListaVar($dec.lista_out);
    	     		}       s=statements[list]	{System.out.println($s.list_out.toString());}
    	;
\end{code_txt}	

\subsection{Bloco \texttt{Statements}}
Neste bloco, o processo passou por actualizar a tabela das variáveis e o estado da tartaruga operação a operação. Por falta de tempo, as operações que envolvessem ciclos e diálogos não 
foram consideradas. O resultado deste processo, poderá ser comprovado na subsecção seguinte.\\

Assim, como já foi dito anteriormente, este bloco começa por receber um objecto que contém uma tabela de variáveis e o estado inicial da tartaruga. Para opções básicas, foi trivial 
actualizar o estado da tartaruga, como se podem ver pelos exemplos seguintes:\\

\begin{code_txt}
//************************* Turtle Statement
turtle_commands[ListaVar list_in]
returns[ListaVar list_out]
	:    step[list_in]		{$list_out = $step.list_out;}
	|    rotate 			{list_in.state += $rotate.value_out; 
	if(list_in.state < 0)list_in.state += 360;
	if(list_in.state >= 360)list_in.state -= 360; $list_out = $list_in;}
	|    mode			{list_in.penup = $mode.state_out; $list_out = list_in;}
	|    dialogue[list_in]		{System.out.println("Dialogos a serem ignorados...");}
	|    loc=location		{Position pes = new Position($loc.x_out,$loc.y_out); 
					list_in.pos = pes; list_out=list_in;} 
	;
\end{code_txt}

A complexidade neste tipo de operações fica na produção referente ao \texttt{rotate}, em que prende o valor da direcção do boneco entre o interválo de 0 a 360 graus. A instrução 
\texttt{step}, dependerá de um valor á frente calculado, daí receber como argumento o atributo que diz respeito ao estado do boneco. Dependerá ainda do tipo de instrução, andar para a 
frente ou para trás. De seguida, mostra-se o pedaço de código que diz respeito a esse problema:

\begin{code_txt}
 step[ListaVar list_in]
returns[ListaVar list_out]
	:    FORWARD     ex1=expression[list_in] 
		{
			if(list_in.state == 0){
				list_in.pos.y += $ex1.value_out; 
			}else if(list_in.state == 90){
				list_in.pos.x += $ex1.value_out;
			}else if(list_in.state == 180){
				list_in.pos.y -= $ex1.value_out;
			}else if(list_in.state == 270){
				list_in.pos.x -= $ex1.value_out;
			}
			$list_out = list_in;
		}
	
	|    BACKWARD     ex=expression[list_in]
		{
			if(list_in.state == 0){
				list_in.pos.y -= $ex.value_out; 
			}else if(list_in.state == 90){
				list_in.pos.x -= $ex.value_out;
			}else if(list_in.state == 180){
				list_in.pos.y += $ex.value_out;
			}else if(list_in.state == 270){
				list_in.pos.x += $ex.value_out;
			}
			$list_out = list_in;
		}
	;
\end{code_txt}

Estas instruções, formam então o conjunto que diz respeito à edição do estado da tartaruga. O resto das produções dizem respeito às operações aritméticas e relacionais, que já foram 
abordadas em exercícios anteriores, e podem-se verificar no apêndice final.

\subsection{Testes e Resultados}
De seguida, apresenta-se um teste e o seu respectivo resultado. Como se pode verificar, a cada operação, o estado da tartaruga é actualizado\\

\textbf{Input}\\

\begin{code_txt}
PROGRAM logolissExemplo {
DECLARATIONS
x = 100 , y -> Integer ;
vector = [1,1000,3] -> Array size 8;
w = TRUE -> Boolean ;
STATEMENTS
FORWARD x
RRIGHT
PEN DOWN
RLEFT
RLEFT
BACKWARD 200/199
Forward vector [1]
GOTO 20, 20
} 
\end{code_txt}

\textbf{Output}\\

\begin{code_txt}
Mr. Turtle:
esta na posicao (0,100)
virado para norte
e para ja, com a caneta levantada

Mr. Turtle:
esta na posicao (0,100)
virado para este
e para ja, com a caneta levantada

Mr. Turtle:
esta na posicao (0,100)
virado para este
e para ja, com caneta pousada

Mr. Turtle:
esta na posicao (0,100)
virado para norte
e para ja, com caneta pousada

Mr. Turtle:
esta na posicao (0,100)
virado para oeste
e para ja, com caneta pousada

Mr. Turtle:
esta na posicao (1,100)
virado para oeste
e para ja, com caneta pousada

Mr. Turtle:
esta na posicao (-999,100)
virado para oeste
e para ja, com caneta pousada

Mr. Turtle:
esta na posicao (20,20)
virado para oeste
e para ja, com caneta pousada

Mr. Turtle:
terminou na posicao (20,20)
virado para oeste
com a caneta pousada 
\end{code_txt}

\newpage
\section{Conclusão e Trabalho Futuro}
Em relação à primeira parte do trabalho, pode concluir que este problema não é trivial de todo. Ainda assim, a solução apresentada ficou próxima da versão final usada, tendo apenas sido 
exigido uns retoques aos seus autores. Fica então como trabalho futuro resolver esses retoques automaticamente.\\

Em relação à segunda parte do trabalho, foi bastante interessante ver as duas abordagens de atributos para resolver este problema, nos dois blocos. No bloco \texttt{Declarations} foram 
usados essencialmente atributos sintetizados e no bloco \texttt{Statements} o uso recaiu sobre os atributos herdados. Fica como trabalho futuro acabar de implementar o uso dos ciclos.\\

Vale a pena ainda referir, que embora se perceba que a se poderia optimizar muito mais as soluções apresentadas, decidiu-se cumprir o paradigma correcto do uso de atributos, conforme 
foi aprendido nas aulas.


\end{document}

