\documentclass[11pt,a4paper]{article}
\usepackage{graphicx}
\usepackage{alltt}
\usepackage[portuges]{babel}
\usepackage[utf8x]{inputenc}
\usepackage{color}
\usepackage{fancyhdr}
\usepackage{listings}
\usepackage{multicol}

\setlength{\textwidth}{16.5cm}
\setlength{\textheight}{24cm}
\setlength{\parindent}{1em}
\setlength{\parskip}{0pt plus 1pt}
\setlength{\oddsidemargin}{0cm}
\setlength{\evensidemargin}{0cm}
\setlength{\topmargin}{-1.1cm}
\setlength{\headsep}{20pt}
\setlength{\columnsep}{1.5pc}
\setlength\columnseprule{.4pt}
\setlength\premulticols{6\baselineskip}
\pagestyle{fancy}

\definecolor{castanho_ulisses}{rgb}{0.71,0.33,0.14}
\definecolor{gray_ulisses}{gray}{0.55}
\definecolor{green_ulises}{rgb}{0.2,0.75,0}

\lstdefinelanguage{C_ulisses}
{
basicstyle=\ttfamily\scriptsize,
%backgroundcolor=\color{yellow},
% frameshape={RYRYNYYYY}{yny}{yny}{RYRYNYYYY}, %contornos... muito nice...
sensitive=true,
morecomment=[l][\color{gray_ulisses}\scriptsize]{//},
morecomment=[s][\color{gray_ulisses}\scriptsize]{/*}{*/},
morestring=[b]",
morestring=[b]',
stringstyle=\color{red},
showstringspaces=false,
% firstnumber=\thelstnumber,
numbers=left,
numberstyle=\tiny,
numberblanklines=true,
showspaces=false,
breaklines=true,
showtabs=false,
xleftmargin=-10pt,
xrightmargin=-20pt,
%% funcoes predefinidas...
emph=
{[1]
return,printf,fprintf,if,else,scanf,sscanf,sprintf,malloc,
calloc,realloc,fgets,fputs,puts,system,strcmp,strstr,
strchr,exit,for,while,fclose,fopen,atoi,sizeof,sin,cos
%% e muitos mais que a medida que sejam necessrios irei adicionar...
},
emphstyle={[1]\color{blue}\textbf},
%% tipos de dados e nao so... >mais enph's<;
emph={[2]char,float,double,void,NULL,int,bool},
emphstyle={[2]\color{green_ulises}\textbf},
emph={[3]typedef, struct},
emphstyle={[3]\color{castanho_ulisses}\textbf}
}

\lstdefinelanguage{txt}
{
       basicstyle=\ttfamily\scriptsize,
       showstringspaces=false,
		breaklines=true,
       numbers=left,
       numberstyle=\tiny,
       numberblanklines=true,
       showspaces=false,
       showtabs=false
}
\lstnewenvironment{code_txt}{\lstset{language=txt}}{}
\lstnewenvironment{code_c}{\lstset{language=C_ulisses}}{}

\lstnewenvironment{code_lex}
{\textbf{Código Lex} \hspace{1cm} \hrulefill \lstset{language=txt}}
{\hrule\smallskip}

\title{\sf Engenharia de Linguagens \\
\begin{tabular}{c}
    %\includegraphics[width=.1\textwidth]{stuff/uminho.jpg}
    %\includegraphics[width=.07\textwidth]{stuff/informatica.jpg}\\
    {\small Universidade do Minho}, {\small LEI}\\
    {\small Ano lectivo 2010/2011}\\
    {\small Engenharia Gramatical - Exercício 3}\\
\end{tabular}
}
\author{
    {\small José Pedro Silva - pg17628} \and
    {\small Mário Ulisses Costa - pg15817} \and
    {\small Pedro Faria - pg17684}}
\date{{\small \today}}

\begin{document}

\maketitle

\begin{abstract}

Descrição da solução à proposta para os problemas do dia 06-12-2010
\end{abstract}

\tableofcontents

\section{Problema}
Considere a linguagem para descrever uma Factura. Sabe-se que uma Factura é composta por um cabeçalho e um corpo, e este é composto por um ou mais movimentos.

A GIC abaixo define formalmente uma primeira versão da linguagem Factura, de acordo com a descrição acima:

\begin{code_txt}
      T = { id, str, num}
      N = { Lisp, SExp, SExplist }
      S = Lisp
      P = {
            p1:  Factura  --> Cabec Corpo
            p2:  Cabec    --> IdFact IdE IdR
            p3:  IdFact   --> NumFact
            p4:  NumFact  --> id
            p5:  IdE      --> Nome NIF Morada NIB
            p6:  IdR      --> Nome NIF Morada
            p7:  Nome     --> str
            p8:  NIF      --> str
            p9:  Morada   --> str
            p10: NIB      --> str
            p11: Corpo    --> ...
          }
\end{code_txt}

Pede-se então que escreva uma Gramática de Atributos, GA, para
\begin{itemize}
    \item a) calcular o total por linha e total geral. 

    \item b) estender a linguagem original para permitir mais do que uma factura (calculando os mesmos totais). 

    \item c) modificar a linguagem de modo a suportar inicialmente a descrição do stock (referência, descrição, preço unitário e quantidade em stock); neste caso, cada linha só terá a referência e a quantidade vendida. 

    \item d) estender a semântica da nova linguagem de modo a também actualizar o stock. 
\end{itemize}

\section{Solução}
As subsecções que se seguem referem-se à resolução de cada alínea do capítulo anterior.
\subsection{Alínea a)}
A solução para a primeira alínea passa por calcular o total por linha na produção linha, multiplicando a quantidade pelo preço unitário.

\begin{code_txt} 
Linha -> Ref Desc Qtd PU {Linha.totalLinha = Qtd.quantidade * PU.pu;}
\end{code_txt}

Para calcular o montante total da factura, temos que somar na produção Linhas, o preço de cada linha com o preço já calculado das restantes linhas.
No caso de só haver uma linha, o montante total é igual ao montante da linha.

\begin{code_txt} 
Linhas -> Linha {Linhas.total = Linhas.totalLinha;}
Linhas -> Linha Linhas {Linhas0.total = Linha1.totalLinha + Linhas2.total;}
\end{code_txt}

\subsection{Alínea b)}
Para a linguagem passar a permitir multiplas facturas, temos de acrescentar as seguintes produções à gramática:

\begin{code_txt} 
LivroF -> Facturas

Facturas -> Factura

Facturas -> Factura Facturas
\end{code_txt}
 
E agora \textit{LivroF} toma o lugar de símbolo inícial, que pertencia a \textit{Factura}.
 
\subsection{Alínea c)}
Nesta alínea a descrição e preço unitário desaparecem de cada linha da factura, porque passa a existir um stock, que contém as informações de todos os produtos.

\begin{code_txt} 
Inicial -> Stock LivroF { LivroF.stock = Stock.stock; }

Stock -> Produtos

Produtos -> Produto

Produtos -> Produto Produtos

Produto -> Ref Desc Qtd PU
\end{code_txt}

As informações lidas no stock são depois herdadas pelo livro de facturas (\textit{LivroF}), depois pelas Facturas, e assim sucessivamente atée chegar à Linha.
Na linha é calculado o preço total da linha através da quantidade e do preço unitário, tal como anteriormente. A diferença é que agora o preço unitário é acessido através do stock.

\begin{code_txt} 
Linha -> Ref Qtd {  
	int pu = Linha.stock.getPU(Ref.num);
	Linha.totalLinha = Qtd.quantidade * pu;
	imprime Linha.totalLinha;
}
\end{code_txt} 


\subsection{Alínea d)}
De modo a actualizar o stock, na produção \textit{Linha} retiramos ao produto referenciado por \textit{Ref}, a quantidade vendida.

\begin{code_txt} 
Linha -> Ref Qtd {  
		int pu = stock.getPU(Ref.num);
 		Linha.totalLinha = Qtd.quantidade * pu;
		imprime Linha.totalLinha;
		
		int qt = Linha.stock.getQuantidade(Ref.num);
		Linha.stock.setQuantidade(qt-Qtd.quantidade);	
		}
\end{code_txt} 

\newpage

\section{Anexo}
\subsection{Código da alínea a)}
\begin{code_txt} 
synthesided attribute quantidade occurs on Qtd;
synthesided attribute pu occurs on PU;
synthesided attribute totalLinha occurs on Linha;
synthesided attribute total occurs on Linhas,CorpoF,Factura,Facturas;


Factura -> CabecaF CorpoF {imprime(CorpoF.total);}

CabecaF -> IdFact IdE IdR

IdFact -> NumFact

NumFact -> id

IdE -> Nome NIF Mor NIB

IdR -> Nome NIF Mor

Nome -> string

NIF -> num

Mor -> string

NIB -> num

CorpoF -> Linhas {CorpoF.total = Linhas.total;}

Linhas -> Linha {Linhas.total = Linhas.totalLinha;}

Linhas -> Linha Linhas {Linhas0.total = Linha1.totalLinha + Linhas2.total;}

Linha -> Ref Desc Qtd PU {Linha.totalLinha = Qtd.quantidade * PU.pu;}

Ref -> NumProd

NumPro -> id

Desc -> string

Qtd -> num  { Qtd.quantidade = num.lexeme;}

PU -> num	{ PU.quantidade = num.lexeme;}




\end{code_txt} 
\newpage
\subsection{Código da alínea b)}
\begin{code_txt} 
synthesided attribute quantidade occurs on Qtd;
synthesided attribute pu occurs on PU;

synthesided attribute totalLinha occurs on Linha;

synthesided attribute total occurs on Linhas,CorpoF,Factura;



LivroF -> Facturas

Facturas -> Factura

Facturas -> Factura Facturas

Factura -> CabecaF CorpoF {imprime(CorpoF.total);}

CabecaF -> IdFact IdE IdR

IdFact -> NumFact

NumFact -> id

IdE -> Nome NIF Mor NIB

IdR -> Nome NIF Mor

Nome -> string

NIF -> num

Mor -> string

NIB -> num

CorpoF -> Linhas {CorpoF.total = Linhas.total;}

Linhas -> Linha {Linhas.total = Linhas.totalLinha;}

Linhas -> Linha Linhas {Linhas0.total = Linha1.totalLinhas + Linhas2.total;}

Linha -> Ref Desc Qtd PU {Linha.totalLinha = Qtd.quantidade * PU.pu;}

Ref -> NumProd

NumPro -> id

Desc -> string

Qtd -> num  { Qtd.quantidade = num.lexeme;}

PU -> num	{ PU.quantidade = num.lexeme;}




\end{code_txt} 
\newpage
\subsection{Código da alínea c)}
\begin{code_txt} 
synthesided attribute quantidade occurs on Qtd;
synthesided attribute pu occurs on PU;

synthesided attribute totalLinha occurs on Linha;

synthesided attribute total occurs on Linhas,CorpoF,Factura,Facturas;

inherited attribute stock occurs on LivroF,Facturas,Factura,CorpoF,Linhas,Linha;



Inicial -> Stock LivroF { LivroF.stock = Stock.stock; }

Stock -> Produtos

Produtos -> Produto

Produtos -> Produto Produtos

Produto -> Ref Desc Qtd PU

LivroF -> Facturas { Facturas.stock = LivroF.stock;}

Facturas -> Factura {Factura.stock = Facturas.stock;}

Facturas -> Factura Facturas {Factura1.stock = Facturas0.stock; Facturas2.stock = Facturas0.stock;} 

Factura -> CabecaF CorpoF { imprime(CorpoF.total); CorpoF.stock = Factura.stock;}

CabecaF -> IdFact IdE IdR

IdFact -> NumFact

NumFact -> id

IdE -> Nome NIF Mor NIB

IdR -> Nome NIF Mor

Nome -> string

NIF -> num

Mor -> string

NIB -> num

CorpoF -> Linhas {CorpoF.total = Linhas.total; Linhas.stock = CorpoF.stock;} 

Linhas -> Linha {Linhas.total = Linhas.totalLinha; Linha.stock = Linhas.stock;}

Linhas -> Linha Linhas {Linhas0.total = Linha1.totalLinhas + Linhas2.total; Linha1.stock = Linhas0.stock; Linhas2.stock = Linhas0.stock; } 

Linha -> Ref Qtd {  int pu = Linha.stock.getPU(Ref.num);
 					Linha.totalLinha = Qtd.quantidade * pu;
					imprime Linha.totalLinha;
						}

Ref -> NumProd { Ref.num = Numprod.num}

NumPro -> id {NumPro.num = id.lexeme}

Desc -> string

Qtd -> num  { Qtd.quantidade = num.lexeme;}

PU -> num	{ PU.quantidade = num.lexeme;}




\end{code_txt} 
\newpage
\subsection{Código da alínea d)}
\begin{code_txt} 
synthesided attribute quantidade occurs on Qtd;
synthesided attribute pu occurs on PU;

synthesided attribute totalLinha occurs on Linha;

synthesided attribute total occurs on Linhas,CorpoF,Factura;

inherited attribute stock occurs on LivroF,Facturas,Factura,CorpoF,Linhas,Linha;




Inicial -> Stock LivroF { LivroF.stock = Stock.stock; }

LivroF -> Facturas { Facturas.stock = LivroF.stock;}

Facturas -> Factura {Factura.stock = Facturas.stock;}

Facturas -> Factura Facturas {Factura1.stock = Facturas0.stock; Facturas2.stock = Facturas0.stock;} 

Factura -> CabecaF CorpoF { imprime(CorpoF.total); CorpoF.stock = Factura.stock;}

CabecaF -> IdFact IdE IdR

IdFact -> NumFact

NumFact -> id

IdE -> Nome NIF Mor NIB

IdR -> Nome NIF Mor

Nome -> string

NIF -> num

Mor -> string

NIB -> num

CorpoF -> Linhas {CorpoF.total = Linhas.total; Linhas.stock = CorpoF.stock;} 

Linhas -> Linha {Linhas.total = Linhas.totalLinha; Linha.stock = Linhas.stock;}

Linhas -> Linha Linhas {Linhas0.total = Linha1.totalLinhas + Linhas2.total; Linha1.stock = Linhas0.stock; Linhas2.stock = Linhas0.stock; } 

Linha -> Ref Qtd {  int pu = stock.getPU(Ref.num);
 					Linha.totalLinha = Qtd.quantidade * pu;
					imprime Linha.totalLinha;
					
					int qt = Linha.stock.getQuantidade(Ref.num);
					Linha.stock.setQuantidade(qt-Qtd.quantidade);	
						}

Ref -> NumProd { Ref.num = Numprod.num}

NumPro -> id {NumPro.num = id.lexeme}

Desc -> string

Qtd -> num  { Qtd.quantidade = num.lexeme;}

PU -> num	{ PU.quantidade = num.lexeme;}



\end{code_txt} 

\end{document}

