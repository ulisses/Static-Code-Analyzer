\documentclass[11pt,a4paper]{article}
\usepackage{graphicx}
\usepackage{alltt}
\usepackage[portuges]{babel}
\usepackage[utf8x]{inputenc}
\usepackage{color}
\usepackage{fancyhdr}
\usepackage{listings}
\usepackage{multicol}

\setlength{\textwidth}{16.5cm}
\setlength{\textheight}{24cm}
\setlength{\parindent}{1em}
\setlength{\parskip}{0pt plus 1pt}
\setlength{\oddsidemargin}{0cm}
\setlength{\evensidemargin}{0cm}
\setlength{\topmargin}{-1.1cm}
\setlength{\headsep}{20pt}
\setlength{\columnsep}{1.5pc}
\setlength\columnseprule{.4pt}
\setlength\premulticols{6\baselineskip}
\pagestyle{fancy}

\definecolor{castanho_ulisses}{rgb}{0.71,0.33,0.14}
\definecolor{gray_ulisses}{gray}{0.55}
\definecolor{green_ulises}{rgb}{0.2,0.75,0}

\lstdefinelanguage{C_ulisses}
{
basicstyle=\ttfamily\scriptsize,
%backgroundcolor=\color{yellow},
% frameshape={RYRYNYYYY}{yny}{yny}{RYRYNYYYY}, %contornos... muito nice...
sensitive=true,
morecomment=[l][\color{gray_ulisses}\scriptsize]{//},
morecomment=[s][\color{gray_ulisses}\scriptsize]{/*}{*/},
morestring=[b]",
morestring=[b]',
stringstyle=\color{red},
showstringspaces=false,
% firstnumber=\thelstnumber,
numbers=left,
numberstyle=\tiny,
numberblanklines=true,
showspaces=false,
breaklines=true,
showtabs=false,
xleftmargin=-10pt,
xrightmargin=-20pt,
%% funcoes predefinidas...
emph=
{[1]
return,printf,fprintf,if,else,scanf,sscanf,sprintf,malloc,
calloc,realloc,fgets,fputs,puts,system,strcmp,strstr,
strchr,exit,for,while,fclose,fopen,atoi,sizeof,sin,cos
%% e muitos mais que a medida que sejam necessrios irei adicionar...
},
emphstyle={[1]\color{blue}\textbf},
%% tipos de dados e nao so... >mais enph's<;
emph={[2]char,float,double,void,NULL,int,bool},
emphstyle={[2]\color{green_ulises}\textbf},
emph={[3]typedef, struct},
emphstyle={[3]\color{castanho_ulisses}\textbf}
}

\lstdefinelanguage{txt}
{
       basicstyle=\ttfamily\scriptsize,
       showstringspaces=false,
       breaklines=true,
       numbers=left,
       numberstyle=\tiny,
       numberblanklines=true,
       showspaces=false,
       showtabs=false
}
\lstnewenvironment{code_txt}{\lstset{language=txt}}{}
\lstnewenvironment{code_c}{\lstset{language=C_ulisses}}{}

\lstnewenvironment{code_lex}
{\textbf{Código Lex} \hspace{1cm} \hrulefill \lstset{language=txt}}
{\hrule\smallskip}

\title{\sf Engenharia de Linguagens \\
\begin{tabular}{c}
    %\includegraphics[width=.1\textwidth]{stuff/uminho.jpg}
    %\includegraphics[width=.07\textwidth]{stuff/informatica.jpg}\\
    {\small Universidade do Minho}, {\small LEI}\\
    {\small Ano lectivo 2010/2011}\\
    {\small Engenharia Gramatical - Exercício 3}\\
\end{tabular}
}
\author{
    {\small José Pedro Silva - pg17628} \and
    {\small Mário Ulisses Costa - pg15817} \and
    {\small Pedro Faria - pg17684}}
\date{{\small \today}}

\begin{document}

\maketitle

\begin{abstract}

Descrição da solução à proposta para os problemas do dia 06-12-2010
\end{abstract}

\tableofcontents

\section{Problema}
Considere a linguagem para descrever uma Factura. Sabe-se que uma Factura é composta por um cabeçalho e um corpo, e este é composto por um ou mais movimentos.

A GIC abaixo define formalmente uma primeira versão da linguagem Factura, de acordo com a descrição acima:

\begin{code_txt}
      T = { id, str, num}
      N = { Lisp, SExp, SExplist }
      S = Lisp
      P = {
            p1: Factura --> Cabec Corpo
            p2: Cabec --> IdFact IdE IdR
            p3: IdFact --> NumFact
            p4: NumFact --> id
            p5: IdE --> Nome NIF Morada NIB
            p6: IdR --> Nome NIF Morada
            p7: Nome --> str
            p8: NIF --> str
            p9: Morada --> str
            p10: NIB --> str
            p11: Corpo --> ...
          }
\end{code_txt}

Pede-se então que escreva uma Gramática de Atributos, GA, para
\begin{itemize}
    \item a) calcular o total por linha e total geral.

    \item b) estender a linguagem original para permitir mais do que uma factura (calculando os mesmos totais).

    \item c) modificar a linguagem de modo a suportar inicialmente a descrição do stock (referência, descrição, preço unitário e quantidade em stock); neste caso, cada linha só terá a referência e a quantidade vendida.

    \item d) estender a semântica da nova linguagem de modo a também actualizar o stock.
\end{itemize}

\section{Solução}
As subsecções que se seguem referem-se à resolução de cada alínea do capítulo anterior.
\subsection{Alínea a)}
A solução para a primeira alínea passa por calcular o total por linha na produção linha, multiplicando a quantidade pelo preço unitário.

\begin{code_txt}
Linha -> Ref Desc Qtd PU {Linha.totalLinha = Qtd.quantidade * PU.pu;}
\end{code_txt}

Para calcular o montante total da factura, temos que somar na produção Linhas, o preço de cada linha com o preço já calculado das restantes linhas.
No caso de só haver uma linha, o montante total é igual ao montante da linha.

\begin{code_txt}
Linhas -> Linha {Linhas.total = Linhas.totalLinha;}
Linhas -> Linha Linhas {Linhas0.total = Linha1.totalLinha + Linhas2.total;}
\end{code_txt}

\subsection{Alínea b)}
Para a linguagem passar a permitir multiplas facturas, temos de acrescentar as seguintes produções à gramática:

\begin{code_txt}
LivroF -> Facturas

Facturas -> Factura

Facturas -> Factura Facturas
\end{code_txt}
 
E agora \textit{LivroF} toma o lugar de símbolo inícial, que pertencia a \textit{Factura}.
 
\subsection{Alínea c)}
Nesta alínea a descrição e preço unitário desaparecem de cada linha da factura, porque passa a existir um stock, que contém as informações de todos os produtos.

\begin{code_txt}
Inicial -> Stock LivroF { LivroF.stock = Stock.stock; }

Stock -> Produtos

Produtos -> Produto

Produtos -> Produto Produtos

Produto -> Ref Desc Qtd PU
\end{code_txt}

As informações lidas no stock são depois herdadas pelo livro de facturas (\textit{LivroF}), depois pelas Facturas, e assim sucessivamente atée chegar à Linha.
Na linha é calculado o preço total da linha através da quantidade e do preço unitário, tal como anteriormente. A diferença é que agora o preço unitário é acessido através do stock.

\begin{code_txt}
Linha -> Ref Qtd {
int pu = Linha.stock.getPU(Ref.num);
Linha.totalLinha = Qtd.quantidade * pu;
imprime Linha.totalLinha;
}
\end{code_txt}


\subsection{Alínea d)}
De modo a actualizar o stock, na produção \textit{Linha} retiramos ao produto referenciado por \textit{Ref}, a quantidade vendida.

\begin{code_txt}
Linha -> Ref Qtd {
int pu = stock.getPU(Ref.num);
  Linha.totalLinha = Qtd.quantidade * pu;
imprime Linha.totalLinha;

int qt = Linha.stock.getQuantidade(Ref.num);
Linha.stock.setQuantidade(qt-Qtd.quantidade);
}
\end{code_txt}

\newpage

\section{AntLR}
Nesta secção será apresentado um estudo sobre a ferramenta \emph{AntLR} usando como caso de estudo 
o problema e a solução proposta nas secções anteriores. Assim, começar-se-à por dar um exemplo de input da linguagem, a gramática desenvolvida 
e a semântica associada.
\subsection{Exemplo de input}

O texto seguinte representa um exemplo de frases no qual a linguagem foi baseada. Cada frase será constituída por uma factura, em que esta contém 
um \emph{id} associado, dados sobre o emissor e receptor, e uma lista de produtos.

\begin{code_txt}
 @idfact:ajf84R5dd5
	@ide:
		@nome:Pedro
		@nif:1234512345
		@morada:Av. D. Joao IV 175
		@nib:13351342
	@idr:
		@nome:Maria
		@nif:1234612345
		@morada:Urbanizacao Salgueiral
	@corpo
		@p:
		g73cc9d90
		2
		10
		descricao ..
		@p:
		g23ac9gg0
		3
		1
		descricao2 ..
	@fimcorpo

@idfact:zf8XXu80e
	@ide:
		@nome:Josuef
		@nif:1278945621
		@morada:Avenida dos Aliados
		@nib:13377597
	@idr:
		@nome:Maria
		@nif:1234612345
		@morada:Urbanizacao Salgueiral
	@corpo
		@p:
		g73cc9d90
		2
		50
		descricao3 ..
	@fimcorpo
\end{code_txt}

Como objectivo final, pretendia-se implementar um \emph{pretty print} da informação mais o número de produtos e custo totais. 

\subsection{Tokens}

Como se pode ver no exemplo anterior, existem certas linhas com \emph{tags} no começo. Essas linhas serviram apenas como auxílio para \emph{debug} da 
linguagem e para explorar a definição de \emph{tokens} da ferramenta. Sendo assim, desenvolveram-se as seguintes \emph{tags}:

\begin{code_txt}
 
tokens{
	TAGidfact = '@idfact:';
	TAGide = '@ide:';
	TAGnome = '@nome:';
	TAGnif = '@nif:';
	TAGmorada = '@morada:';
	TAGnib = '@nib:';
	TAGidr = '@idr:';
	TAGcorpo = '@corpo';
	TAGfimcorpo = '@fimcorpo';
	TAGproduto = '@p:';
}

\end{code_txt}

\subsection{Gramática sem semântica}

De seguida, apresenta-se a gramática criada com a \emph{syntax} do \emph{AntLR}. A explicação segue a que foi dada nas secções anteriores, com a 
excepção de não ter sido criado um Stock para esta linguagem. Como se pode ver também, foram adicionados os tokens referidos na subsecção anterior à gramática. 

\begin{code_txt}
 
livrofact	:	(factura)+
		;

factura		:	cabec	corpo
		;
	
cabec 		:	idfact 	ide	idr
		;
	
corpo		:	TAGcorpo	(produto)+	TAGfimcorpo
		;	
	
produto 	:	(TAGproduto	referencia 	precouni	quantidade	descricao)
		;	
	
referencia	:	ID
		;	
	
descricao	:	STRINGPLUS
		;	
	
quantidade	:	NUM
		;	
	
precouni	:	NUM
		;	
	
idfact 		:	numfact
		;
	
numfact		:	TAGidfact	ID
		;
	
ide		:	TAGide	nome	nif	morada	nib	
		;
	
idr		:	TAGidr	nome	nif	morada
		;

nome		:	TAGnome	STRING
		;
	
nif		:	TAGnif	NUM
		;

morada		:	TAGmorada	STRINGPLUS
		;

nib		:	TAGnib	NUM
		;

NUM 		:	('0'..'9')+
		;

STRING		:	('a'..'z'|'A'..'Z')+
		;

ID		:	('a'..'z'|'A'..'Z'|'0'..'9')+
		;
	
STRINGPLUS	:	('a'..'z'|'A'..'Z'|'0'..'9'|'\.'|' ')+
		;

NS		:  	(' ' | '\t' | '\n' | 'r') { skip(); }
   		;
\end{code_txt}

Em relação às listas que possam surgir nesta linguagem (de facturas e/ou produtos) foram definidas com uma notação, do estilo expressões regulares,
 que esta ferramenta permite.

\subsection{Semântica}

Para desenvolvimento dos problemas decidiu-se tirar vantagem do \emph{AntLR} ser perfeitamente compatível com \emph{Java}. Assim, um \texttt{livrofact}
 seria um \emph{ArrayList} de objectos \texttt{Factura}. Cada objecto \texttt{Factura} teria 4 variáveis: uma \texttt{String} id; um objecto \texttt{Emissor}; 
um objecto \texttt{Receptor}; e uma \emph{ArrayList} de objectos \texttt{Produto}. \\

Os objectos \texttt{Receptor}, \texttt{Emissor} e \texttt{Produto} apenas teriam informação de tipos primitivos. \\

Assim, no começo da gramática, é inicializado o \emph{ArrayList} de \texttt{Factura}, como se pode ver na linha 2 do pedaço de código seguinte:\\

\begin{code_txt}
livrofact 
@init{ArrayList<Factura> livrof_in = new ArrayList<Factura>();}
	:	(factura {livrof_in.add(\$factura.factura_out);})+ {System.out.println("Livro de Facturas\n"); 
						for(Factura f : livrof_in){System.out.println(f.toString());}}
	;
\end{code_txt}

A cada factura lida, seria guardado um atributo sintetizado do tipo \texttt{Factura} (linha 3). E no final seria imprimida a informação (linha 4).\\

O pedaço de código seguinte, diz respeito à criação de facturas. A cada início da produção \texttt{factura}, é inicializado um objecto \texttt{Factura} 
(linha 3). Esse objecto criado será passado como um atributo herdado para as produções \texttt{cabec} e \texttt{corpo} (linhas 4 e 5). Na produção
 \texttt{cabec}, o atributo herdado \texttt{factura\_in} terá a variável \emph{id} alterada pelo atributo sintetizado vindo do nodo \texttt{idfact} (linha 10).
 Para os nodos \texttt{ide} e \texttt{idr} serão mandados as variáveis \texttt{emi} e \texttt{rec}, os objectos \texttt{Emissor} e \texttt{Receptor} 
respectivamente (linhas 11 e 12). O processamento do corpo da factura ocorre de maneira semelhante ao da criação da linha de facturas, a 
cada ocorrência de um nodo \texttt{produto} seria adicionado à lista de produtos um atributo sintetizado do tipo \texttt{Produto}.

\begin{code_txt}
 factura 
returns[Factura factura_out]
@init{Factura factura_in = new Factura();}
	:	cabec[factura_in] {  }
		corpo[factura_in] { factura_out = factura_in; }
	;
	
cabec [Factura factura_in]

	:	idfact {$factura_in.id = $idfact.id_out;}
		ide[factura_in.emi]
		idr[factura_in.rec]
	;
	
corpo [Factura factura_in]	
	:	TAGcorpo
		(produto{$factura_in.addProduto($produto.prod_out);})+
		TAGfimcorpo
	;
\end{code_txt}

Tanto o Emissor como o Receptor e os Produto têm processos de criação semelhantes. Não serão precisos de ser inicializados objectos dos tipos
 Emissor e Receptor, pois o seu processo ocorre em simultâneo com a inicialização do objecto \texttt{Factura}. Assim, cada variável do 
objecto herdado \texttt{Emissor} fica com os valores sintetizados correspondentes (linhas 3, 4, 5 e 6).

\begin{code_txt}
ide [Emissor emissor_in]
	:	TAGide	
		nome	{$emissor_in.nome = $nome.nome_out;}
		nif	{$emissor_in.nif = $nif.nif_out;}
		morada	{$emissor_in.morada = $morada.morada_out;}
		nib	{$emissor_in.nib = $nib.nib_out;}
	;
\end{code_txt}

Os nodos \texttt{nome}, \texttt{nif}, \texttt{morada} e \texttt{nib} são bastante semelhantes, apenas retornam o valor para os nodos superiores respectivos, como por exemplo:

\begin{code_txt}
 nib
returns[int nib_out]
	:	TAGnib	NUM{$nib_out = Integer.parseInt($NUM.text);}
	;
\end{code_txt}

Todo o restante código semântico será adicionado na seção \texttt{Anexo}.

\newpage
\section{Anexo}
\subsection{Código da alínea a)}
\begin{code_txt}
synthesided attribute quantidade occurs on Qtd;
synthesided attribute pu occurs on PU;
synthesided attribute totalLinha occurs on Linha;
synthesided attribute total occurs on Linhas,CorpoF,Factura,Facturas;


Factura -> CabecaF CorpoF {imprime(CorpoF.total);}

CabecaF -> IdFact IdE IdR

IdFact -> NumFact

NumFact -> id

IdE -> Nome NIF Mor NIB

IdR -> Nome NIF Mor

Nome -> string

NIF -> num

Mor -> string

NIB -> num

CorpoF -> Linhas {CorpoF.total = Linhas.total;}

Linhas -> Linha {Linhas.total = Linhas.totalLinha;}

Linhas -> Linha Linhas {Linhas0.total = Linha1.totalLinha + Linhas2.total;}

Linha -> Ref Desc Qtd PU {Linha.totalLinha = Qtd.quantidade * PU.pu;}

Ref -> NumProd

NumPro -> id

Desc -> string

Qtd -> num { Qtd.quantidade = num.lexeme;}

PU -> num { PU.quantidade = num.lexeme;}




\end{code_txt}
\newpage
\subsection{Código da alínea b)}
\begin{code_txt}
synthesided attribute quantidade occurs on Qtd;
synthesided attribute pu occurs on PU;

synthesided attribute totalLinha occurs on Linha;

synthesided attribute total occurs on Linhas,CorpoF,Factura;



LivroF -> Facturas

Facturas -> Factura

Facturas -> Factura Facturas

Factura -> CabecaF CorpoF {imprime(CorpoF.total);}

CabecaF -> IdFact IdE IdR

IdFact -> NumFact

NumFact -> id

IdE -> Nome NIF Mor NIB

IdR -> Nome NIF Mor

Nome -> string

NIF -> num

Mor -> string

NIB -> num

CorpoF -> Linhas {CorpoF.total = Linhas.total;}

Linhas -> Linha {Linhas.total = Linhas.totalLinha;}

Linhas -> Linha Linhas {Linhas0.total = Linha1.totalLinhas + Linhas2.total;}

Linha -> Ref Desc Qtd PU {Linha.totalLinha = Qtd.quantidade * PU.pu;}

Ref -> NumProd

NumPro -> id

Desc -> string

Qtd -> num { Qtd.quantidade = num.lexeme;}

PU -> num { PU.quantidade = num.lexeme;}




\end{code_txt}
\newpage
\subsection{Código da alínea c)}
\begin{code_txt}
synthesided attribute quantidade occurs on Qtd;
synthesided attribute pu occurs on PU;

synthesided attribute totalLinha occurs on Linha;

synthesided attribute total occurs on Linhas,CorpoF,Factura,Facturas;

inherited attribute stock occurs on LivroF,Facturas,Factura,CorpoF,Linhas,Linha;



Inicial -> Stock LivroF { LivroF.stock = Stock.stock; }

Stock -> Produtos

Produtos -> Produto

Produtos -> Produto Produtos

Produto -> Ref Desc Qtd PU

LivroF -> Facturas { Facturas.stock = LivroF.stock;}

Facturas -> Factura {Factura.stock = Facturas.stock;}

Facturas -> Factura Facturas {Factura1.stock = Facturas0.stock; Facturas2.stock = Facturas0.stock;}

Factura -> CabecaF CorpoF { imprime(CorpoF.total); CorpoF.stock = Factura.stock;}

CabecaF -> IdFact IdE IdR

IdFact -> NumFact

NumFact -> id

IdE -> Nome NIF Mor NIB

IdR -> Nome NIF Mor

Nome -> string

NIF -> num

Mor -> string

NIB -> num

CorpoF -> Linhas {CorpoF.total = Linhas.total; Linhas.stock = CorpoF.stock;}

Linhas -> Linha {Linhas.total = Linhas.totalLinha; Linha.stock = Linhas.stock;}

Linhas -> Linha Linhas {Linhas0.total = Linha1.totalLinhas + Linhas2.total; Linha1.stock = Linhas0.stock; Linhas2.stock = Linhas0.stock; }

Linha -> Ref Qtd { int pu = Linha.stock.getPU(Ref.num);
  Linha.totalLinha = Qtd.quantidade * pu;
imprime Linha.totalLinha;
}

Ref -> NumProd { Ref.num = Numprod.num}

NumPro -> id {NumPro.num = id.lexeme}

Desc -> string

Qtd -> num { Qtd.quantidade = num.lexeme;}

PU -> num { PU.quantidade = num.lexeme;}




\end{code_txt}
\newpage
\subsection{Código da alínea d)}
\begin{code_txt}
synthesided attribute quantidade occurs on Qtd;
synthesided attribute pu occurs on PU;

synthesided attribute totalLinha occurs on Linha;

synthesided attribute total occurs on Linhas,CorpoF,Factura;

inherited attribute stock occurs on LivroF,Facturas,Factura,CorpoF,Linhas,Linha;




Inicial -> Stock LivroF { LivroF.stock = Stock.stock; }

LivroF -> Facturas { Facturas.stock = LivroF.stock;}

Facturas -> Factura {Factura.stock = Facturas.stock;}

Facturas -> Factura Facturas {Factura1.stock = Facturas0.stock; Facturas2.stock = Facturas0.stock;}

Factura -> CabecaF CorpoF { imprime(CorpoF.total); CorpoF.stock = Factura.stock;}

CabecaF -> IdFact IdE IdR

IdFact -> NumFact

NumFact -> id

IdE -> Nome NIF Mor NIB

IdR -> Nome NIF Mor

Nome -> string

NIF -> num

Mor -> string

NIB -> num

CorpoF -> Linhas {CorpoF.total = Linhas.total; Linhas.stock = CorpoF.stock;}

Linhas -> Linha {Linhas.total = Linhas.totalLinha; Linha.stock = Linhas.stock;}

Linhas -> Linha Linhas {Linhas0.total = Linha1.totalLinhas + Linhas2.total; Linha1.stock = Linhas0.stock; Linhas2.stock = Linhas0.stock; }

Linha -> Ref Qtd { int pu = stock.getPU(Ref.num);
  Linha.totalLinha = Qtd.quantidade * pu;
imprime Linha.totalLinha;

int qt = Linha.stock.getQuantidade(Ref.num);
Linha.stock.setQuantidade(qt-Qtd.quantidade);
}

Ref -> NumProd { Ref.num = Numprod.num}

NumPro -> id {NumPro.num = id.lexeme}

Desc -> string

Qtd -> num { Qtd.quantidade = num.lexeme;}

PU -> num { PU.quantidade = num.lexeme;}



\end{code_txt}
\newpage
\subsection{AntLR}
\begin{code_txt}
 grammar livrofact;

tokens{
	TAGidfact = '@idfact:';
	TAGide = '@ide:';
	TAGnome = '@nome:';
	TAGnif = '@nif:';
	TAGmorada = '@morada:';
	TAGnib = '@nib:';
	TAGidr = '@idr:';
	TAGcorpo = '@corpo';
	TAGfimcorpo = '@fimcorpo';
	TAGproduto = '@p:';
}

@header{
	;
}

@members{

/*===========  Classe Emissor ============= */
public class Emissor{

	public String nome;
	public String morada;
	public int nib;
	public int nif;
	
	public Emissor(){
		this.nome = "";
		this.morada = "";
		this.nib = 0;
		this.nif = 0;
	}
	
	public Emissor(String name, String street, int enib, int enif){
		this.nome = name;
		this.morada = street;
		this.nib = enib;
		this.nif = enif;
	}
	
	public String toString(){
		return "\nDados Emissor" + "\nNome: " + this.nome + "\nMorada: " + this.morada + "\n";
		//return "\nEmissor\n";
	}

}

/*===========  Classe Receptor ============= */
public class Receptor{

	public String nome;
	public String morada;
	public int nif;
	
	public Receptor(){
		this.nome = "";
		this.morada = "";
		this.nif = 0;
	}
	
	public Receptor(String name, String street, int enif){
		this.nome = name;
		this.morada = street;
		this.nif = enif;
	}
	
	public String toString(){
		return "\nDados Receptor" + "\nNome: " + this.nome + "\nMorada: " + this.morada + "\n";
		//return "\nReceptor\n";
	}

}

/*===========  Produto ============= */
public class Produto{
	public String referencia;
	public String descricao;
	public int quantidade;
	public int precouni;
	
	public Produto(){
		this.referencia = "";
		this.descricao = "";
		this.precouni = 0;
		this.quantidade = 0;
	}
	
	public Produto(String ref, String desc, int pu, int qtd){
		this.referencia = ref;
		this.descricao = desc;
		this.precouni = pu;
		this.quantidade = qtd;
	}
	
	public String toString(){
		return "Produto -> " + this.referencia + "\nDescricao: " + this.descricao + "\nPreco por unidade: " + 
		this.precouni + "\nQuantidade" + this.quantidade + "\n";
	}
	
	public int precoTotal(){
		return this.precouni * this.quantidade;
	}
}

/*===========  Factura ============= */
public class Factura{
	public Emissor emi;
	public Receptor rec;
	public String id;
	public ArrayList<Produto> al;
	
	public Factura(){
		this.id = "";
		this.emi = new Emissor();
		this.rec = new Receptor();
		this.al = new ArrayList<Produto>();
	}
	
	public Factura(Emissor e, Receptor r){
		this.id = "";
		this.emi = new Emissor(e.nome, e.morada, e.nib, e.nif);
		this.rec = new Receptor(r.nome, r.morada, r.nif);
		this.al = new ArrayList<Produto>();
	}
	
	public void addProduto(Produto p){
		this.al.add(p);
	}
	
	public String toString(){
		int custo = 0;
		for(Produto p : this.al){
			custo += p.precoTotal();
		}
	
		return "\nFactura " + this.id + "\n" +
			this.emi.toString() + 
			this.rec.toString() + 
			"\nTotal de Produtos = " + this.al.size() +
			"\nCusto Total = " + custo;
	}
}

}

livrofact 
@init{ArrayList<Factura> livrof_in = new ArrayList<Factura>();}
	:	(factura {livrof_in.add(\$factura.factura_out);})+ {System.out.println("Livro de Facturas\n"); 
						for(Factura f : livrof_in){System.out.println(f.toString());}}
	;

factura 
returns[Factura factura_out]
@init{Factura factura_in = new Factura();}
	:	cabec[factura_in] {  }//System.out.println(factura_in.toString());}
		corpo[factura_in] { factura_out = factura_in; }
	;
	
cabec [Factura factura_in]

	:	idfact {\$factura_in.id = \$idfact.id_out;}
		ide[factura_in.emi]
		idr[factura_in.rec]
	;
	
corpo [Factura factura_in]	
	:	TAGcorpo
		(produto{\$factura_in.addProduto(\$produto.prod_out);})+
		TAGfimcorpo
	;	
	
produto 
returns[Produto prod_out]
@init{Produto prod_in = new Produto();}	
	:	(TAGproduto	
		referencia {prod_in.referencia = \$referencia.id_out;}
		precouni {prod_in.precouni = \$precouni.pu_out;}
		quantidade {prod_in.quantidade = \$quantidade.qtd_out;} 
		descricao {prod_in.descricao = \$descricao.descri_out;}) {\$prod_out = prod_in;}
		
	;	
	
referencia
returns[String id_out]
	:	ID{\$id_out = \$ID.text;}
	;	
	
descricao
returns[String descri_out]
	:	STRINGPLUS{\$descri_out = \$STRINGPLUS.text;}
	;	
	
quantidade	
returns[int qtd_out]
	:	NUM{\$qtd_out = Integer.parseInt(\$NUM.text);}
	;	
	
precouni
returns[int pu_out]	
	:	NUM{\$pu_out = Integer.parseInt(\$NUM.text);}
	;	
	
	
idfact 
returns[String id_out]
	:	numfact {\$id_out = \$numfact.id_out;}
	;
	
numfact
returns[String id_out]
	:	TAGidfact	ID{\$id_out = \$ID.text;}
	;
	
ide [Emissor emissor_in]

	:	TAGide	
		nome	{\$emissor_in.nome = \$nome.nome_out;}
		nif	{\$emissor_in.nif = \$nif.nif_out;}
		morada	{\$emissor_in.morada = \$morada.morada_out;}
		nib	{\$emissor_in.nib = \$nib.nib_out;}
	;
	
idr [Receptor receptor_in]	

	:	TAGidr	
		nome	{\$receptor_in.nome = \$nome.nome_out;}
		nif	{\$receptor_in.nif = \$nif.nif_out;}
		morada	{\$receptor_in.morada = \$morada.morada_out;}
	;

nome
returns[String nome_out]
	:	TAGnome	STRING{\$nome_out = \$STRING.text;}
	;
	
nif
returns[int nif_out]
	:	TAGnif	NUM{\$nif_out = Integer.parseInt(\$NUM.text);}
	;

morada
returns[String morada_out]
	:	TAGmorada	STRINGPLUS{\$morada_out = \$STRINGPLUS.text;}
	;

nib
returns[int nib_out]
	:	TAGnib	NUM{\$nib_out = Integer.parseInt(\$NUM.text);}
	;

NUM 	:	('0'..'9')+
	;

STRING	:	('a'..'z'|'A'..'Z')+
	;

ID	:	('a'..'z'|'A'..'Z'|'0'..'9')+
	;
	
STRINGPLUS	:	('a'..'z'|'A'..'Z'|'0'..'9'|'\.'|' ')+
	;

NS	:  	(' ' | '\t' | '\n' | 'r') { skip(); }
   	;
\end{code_txt}



\end{document}



