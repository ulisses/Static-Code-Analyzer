\documentclass[11pt,a4paper]{article}
\usepackage{graphicx}
\usepackage{alltt}
\usepackage[portuges]{babel}
\usepackage[utf8x]{inputenc}
\usepackage{color}
\usepackage{fancyhdr}
\usepackage{listings}
\usepackage{multicol}

\setlength{\textwidth}{16.5cm}
\setlength{\textheight}{24cm}
\setlength{\parindent}{1em}
\setlength{\parskip}{0pt plus 1pt}
\setlength{\oddsidemargin}{0cm}
\setlength{\evensidemargin}{0cm}
\setlength{\topmargin}{-1.1cm}
\setlength{\headsep}{20pt}
\setlength{\columnsep}{1.5pc}
\setlength\columnseprule{.4pt}
\setlength\premulticols{6\baselineskip}
\pagestyle{fancy}

\definecolor{castanho_ulisses}{rgb}{0.71,0.33,0.14}
\definecolor{gray_ulisses}{gray}{0.55}
\definecolor{green_ulises}{rgb}{0.2,0.75,0}

\lstdefinelanguage{C_ulisses}
{
basicstyle=\ttfamily\scriptsize,
%backgroundcolor=\color{yellow},
% frameshape={RYRYNYYYY}{yny}{yny}{RYRYNYYYY}, %contornos... muito nice...
sensitive=true,
morecomment=[l][\color{gray_ulisses}\scriptsize]{//},
morecomment=[s][\color{gray_ulisses}\scriptsize]{/*}{*/},
morestring=[b]",
morestring=[b]',
stringstyle=\color{red},
showstringspaces=false,
% firstnumber=\thelstnumber,
numbers=left,
numberstyle=\tiny,
numberblanklines=true,
showspaces=false,
breaklines=true,
showtabs=false,
xleftmargin=-10pt,
xrightmargin=-20pt,
%% funcoes predefinidas...
emph=
{[1]
return,printf,fprintf,if,else,scanf,sscanf,sprintf,malloc,
calloc,realloc,fgets,fputs,puts,system,strcmp,strstr,
strchr,exit,for,while,fclose,fopen,atoi,sizeof,sin,cos
%% e muitos mais que a medida que sejam necessrios irei adicionar...
},
emphstyle={[1]\color{blue}\textbf},
%% tipos de dados e nao so... >mais enph's<;
emph={[2]char,float,double,void,NULL,int,bool},
emphstyle={[2]\color{green_ulises}\textbf},
emph={[3]typedef, struct},
emphstyle={[3]\color{castanho_ulisses}\textbf}
}

\lstdefinelanguage{txt}
{
       basicstyle=\ttfamily\scriptsize,
       showstringspaces=false,
       breaklines=true,
       numbers=left,
       numberstyle=\tiny,
       numberblanklines=true,
       showspaces=false,
       showtabs=false
}
\lstnewenvironment{code_txt}{\lstset{language=txt}}{}
\lstnewenvironment{code_c}{\lstset{language=C_ulisses}}{}

\lstnewenvironment{code_lex}
{\textbf{Código Lex} \hspace{1cm} \hrulefill \lstset{language=txt}}
{\hrule\smallskip}

\title{\sf Engenharia de Linguagens \\
\begin{tabular}{c}
    %\includegraphics[width=.1\textwidth]{stuff/uminho.jpg}
    %\includegraphics[width=.07\textwidth]{stuff/informatica.jpg}\\
    {\small Universidade do Minho}, {\small LEI}\\
    {\small Ano lectivo 2010/2011}\\
    {\small Engenharia Gramatical -Calculadora em ANTLR}\\
\end{tabular}
}
\author{
    {\small José Pedro Silva - pg17628} \and
    {\small Mário Ulisses Costa - pg15817} \and
    {\small Pedro Faria - pg17684}}
\date{{\small \today}}

\begin{document}

\maketitle

\begin{abstract}

Descrição da solução proposta para a gramática de uma calculadora em ANTLR.
\end{abstract}

\tableofcontents

\section{Problema}
Pretende-se escrever uma gramática em ANTLR que possua as produções necessárias para reconhecer uma pequena linguagem que permite:
\begin{itemize}
    \item declarações de variáveis (inteiras e floats)
    \item atribuições de valores a variáveis
    \item operações aritméticas (soma, subtracção, multiplicação e divisão de valores)
\end{itemize}

As acções semânticas, além de calcularem os resultados das operações aritméticas, deverão reportar erros quando:
\begin{itemize}
    \item haja redeclarações de variáveis
    \item atribuições a variáveis não declaradas
    \item atribuir a uma variável tipo inteiro, um float
\end{itemize}


\section{Solução}

\subsection{Estruturas de dados necessárias}
Para guardar os tipos das variáveis já declaradas, assim como os valores que lhes vão sendo atribuidos, criamos um hashmap no qual a key é o nome da variável e o valor é um objecto chamado \textit{Par}.
O \textit{Par} contém duas variáveis privadas, o tipo que é uma String e o value que é um float.
\begin{code_txt}
public class Par{
	private String tipo;
	private float value;
	
	public Par(){
		this.tipo = "";
		this.value = 0;
	}
	
	public Par(String tipo, float value){
		this.tipo = tipo;
		this.value = value;
	}
	
	public String getTipo(){return this.tipo;}	
	public float getValue(){return this.value;}
	
	public void setTipo(String tipo){this.tipo = tipo;}	
	public void setValue(float val){this.value = val;}
}
\end{code_txt}

Também criamos um enum Tipo, que é utilizado para guardar o resultado da avaliação de um valor numérico, em relação ao seu tipo: 
\begin{code_txt}
enum Tipo {REAL, INT};
.
.
.
.
if ($expression.valor == Math.round($expression.valor)) {
	tipo = Tipo.INT;
} else {
	tipo = Tipo.REAL;
	}
\end{code_txt}



\subsection{Descrição da solução}
O hashmap \textit{declarations\_in}, é espalhado por todos os elementos da árvore, quer seja nas declarações para que seja adicionada 
a variável e o seu tipo ao hashmap, quer nas atribuições, nas quais o valor da variável pode ser alterado.
O hashmap \textit{declarations\_out}, que recebe o estado do \textit{declarations\_in} no fim da acção semântica, sobe na árvore, de modo
a ser disponibilizado já actualizado para a próxima declaração ou atribuição.\\
\\
Na produção \textbf{declaration} é verificado se a variável que se pretende declarar ainda não existe, se já existir emite um erro.


\begin{code_txt}
//INPUT:
int x;
real x;
x  = 2;

//OUTPUT
ERRO: A variavel x ja foi declarada antes.
\end{code_txt}


Na produção \textbf{attrib} é verificado se a variável à qual se pretende atribuir um valor, já foi previamente declarada.
\begin{code_txt}
//INPUT:
int x;
y  = 2;


//OUTPUT
ERRO: A variavel y ainda nao foi declarada!
\end{code_txt}

Ainda na produção \textbf{attrib} é feita a verificação do tipo do valor que irá a ser atribuido à variável. Se for fracionário, e se estiver a ser atribuido a um inteiro é emitido um warning.
\begin{code_txt}
//INPUT:
int x;
real y;
x = 1;
x = x div 3;


//OUTPUT
x == 1.0
Warning: Possivel perda de precisao! (guardar real em variavel inteira)
x == 0.0
\end{code_txt}

Nas produções \textbf{expression} e \textbf{term}, vão sendo efectuados os cálculos, quer sejam somas, subtracções, multiplicações ou divisões. O resultado é guardado na produção \textbf{attrib}.

\begin{code_txt}
//INPUT:
int x;
real y;
x = 4 - 6;
y = 4 div 2 * 3 + x;

//OUTPUT
x == -2.0
y == 4.0
\end{code_txt}

\section{Anexos}
\subsection{Calculadora}

\begin{code_txt}
grammar calc_better;

@header{
import java.util.HashMap;
import java.util.ArrayList;
}


@members{
enum Tipo {REAL, INT};

//*===========  Par ============= */
public class Par{
	private String tipo;
	private float value;
	
	public Par(){
		this.tipo = "";
		this.value = 0;
	}
	
	public Par(String tipo, float value){
		this.tipo = tipo;
		this.value = value;
	}
	
	public String getTipo(){return this.tipo;}	
	public float getValue(){return this.value;}
	
	public void setTipo(String tipo){this.tipo = tipo;}	
	public void setValue(float val){this.value = val;}
}

}



language returns [HashMap<String,Par> identificadores]
	 @init {HashMap <String,Par>declarations_in = new HashMap<String,Par>();}
	:	 ((declaration[declarations_in]{declarations_in=$declaration.declarations_out;} 
		  |attrib [declarations_in] {declarations_in=$declaration.declarations_out;} 
		  | io[declarations_in]) ';' )*
		  		{$identificadores = $declaration.declarations_out;}
	;
	
declaration [HashMap<String,Par> declarations_in] returns [HashMap<String,Par> declarations_out]
	:	(tipo=Type_Int | tipo=Type_Real)   ID {
						if (declarations_in.containsKey($ID.text)){
							System.out.println( "ERRO: A variavel " + $ID.text +" ja foi declarada antes.\n");
							
							}
						else	{
							Par p = new Par($tipo.text,0);
					 		 declarations_in.put($ID.text, p);
							}
							declarations_out = declarations_in;
						}
	;

attrib	[HashMap<String,Par> declarations_in] returns [HashMap<String,Par> declarations_out]
	:	ID '=' expression [declarations_in] 
			{
				if (!declarations_in.containsKey($ID.text)){
					 System.out.println("ERRO: A variavel "+ $ID.text + " ainda nao foi declarada!\n" );
				}
				else {
					Tipo tipo;
					if ($expression.valor == Math.round($expression.valor)) {
 					   tipo = Tipo.INT;
					} else {
					    tipo = Tipo.REAL;
					}
					if (declarations_in.get($ID.text).getTipo().equals("int") && tipo == Tipo.REAL ){
						System.out.println( "Warning: Possivel perda de precisao! (guardar real em variavel inteira)");
						Par p = declarations_in.get($ID.text);
						p.setValue(Math.round($expression.valor));
					}
					else{
						Par p = declarations_in.get($ID.text);
						p.setValue($expression.valor);
					}	
					System.out.println($ID.text + " == " + declarations_in.get($ID.text).getValue());
 			 		declarations_out = declarations_in;
				}
			}
	;
	 
io	[HashMap<String,Par> declarations_in]
	:	'?' ID {}
	|	'!' expression [declarations_in]
	;
	
expression [HashMap<String,Par> declarations_in] 
	returns [Tipo tipo, float valor]
	: 
	|	t1=term[declarations_in]{$valor = $t1.valor;} (operatorAdd t2=term[declarations_in]{
									if($operatorAdd.text.equals("+"))
										$valor = $valor + $t2.valor;
									else
						 				$valor = $valor - $t2.valor;
						 			})*  {}		
	;
	
term	[HashMap<String,Par> declarations_in]
	returns [float valor]
	:	f1=factor[declarations_in]{$valor = $f1.valor;} (operatorMul f2=factor[declarations_in]{
									if($operatorMul.text.equals("*"))																	
										$valor = $valor * $f2.valor;
									else
										$valor = $valor / $f2.valor;
										})* {}  
	;
	
factor	[HashMap<String,Par> declarations_in]
	returns [float valor]
	:	'(' expression[declarations_in] ')' {$valor = $expression.valor;}
	| 	value[declarations_in] {$valor = $value.valor;} 
	;
	
operatorAdd
	:	'+' | '-'
	;
	
operatorMul
	:	'*' | 'div'
	;
	
operator:	'+' | '-' | '*' | 'div'
	;
	

value  	[HashMap<String,Par> declarations_in]
	returns[float valor]
 	:	ID {$valor = declarations_in.get($ID.text).getValue();} 
 		| INT {$valor = Integer.parseInt($INT.text);}
 		| FLOAT {$valor = Float.parseFloat($FLOAT.text);}
 	; 
 
Type_Int:	 'int'
	;
		
Type_Real
	:	'real'	 
	;


ID  :	('a'..'z'|'A'..'Z'|'_') ('a'..'z'|'A'..'Z'|'0'..'9'|'_')*
    ;

INT :	'0'..'9'+
    ;

FLOAT
    :   ('0'..'9')+ '.' ('0'..'9')* EXPONENT?
    |   '.' ('0'..'9')+ EXPONENT?
    |   ('0'..'9')+ EXPONENT
    ;

COMMENT
    :   '//' ~('\n'|'\r')* '\r'? '\n' {$channel=HIDDEN;}
    |   '/*' ( options {greedy=false;} : . )* '*/' {$channel=HIDDEN;}
    ;

WS  :   ( ' '
        | '\t'
        | '\r'
        | '\n'
        ) {$channel=HIDDEN;} 
    ;

fragment
EXPONENT : ('e'|'E') ('+'|'-')? ('0'..'9')+ ;


\end{code_txt}
\end{document}



