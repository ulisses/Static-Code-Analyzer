\documentclass[11pt,a4paper,notitlepage]{article}
\usepackage{graphics}
\usepackage{alltt}
\usepackage[portuges]{babel}
\usepackage[utf8x]{inputenc}
\usepackage{color}
\usepackage{fancyhdr}
\usepackage{listings}
\usepackage{multicol}
\usepackage{babel}
\usepackage{hyperref}
\usepackage{t1enc}
%\usepackage{aeguill}
\usepackage{varioref}
\usepackage{fancyvrb}
\usepackage{color}
\usepackage{cite}
\usepackage{float}
\usepackage{listings}
\usepackage{harvard}
\usepackage{geometry}
\usepackage{setspace}
\usepackage{thumbpdf}
\usepackage{fancyhdr}
\usepackage{lastpage}
\usepackage{alltt}
\usepackage{graphicx}
\usepackage{verbatim}
\usepackage{color}
\usepackage{url}
\usepackage{epsf}
\usepackage{listings}
\usepackage[refpages]{gloss}
\usepackage{array}
\usepackage{longtable}
\usepackage{multirow}
\usepackage{amsmath, amsthm, amssymb}
\usepackage{slashbox}
\usepackage{rotating}

\setlength{\textwidth}{16.5cm}
\setlength{\textheight}{24cm}
\setlength{\parindent}{1em}
\setlength{\parskip}{0pt plus 1pt}
\setlength{\oddsidemargin}{0cm}
\setlength{\evensidemargin}{0cm}
\setlength{\topmargin}{-1.1cm}
\setlength{\headsep}{20pt}
\setlength{\columnsep}{1.5pc}
\setlength\columnseprule{.4pt}
\setlength\premulticols{6\baselineskip}
\pagestyle{fancy}

\definecolor{castanho_ulisses}{rgb}{0.71,0.33,0.14}
\definecolor{gray_ulisses}{gray}{0.55}
\definecolor{green_ulises}{rgb}{0.2,0.75,0}

\lstdefinelanguage{C_ulisses}{
	basicstyle=\ttfamily\scriptsize,
    sensitive=true,
	morecomment=[l][\color{gray_ulisses}\scriptsize]{//},
	morecomment=[s][\color{gray_ulisses}\scriptsize]{/*}{*/},
	morestring=[b]",
	morestring=[b]',
	stringstyle=\color{red},
	showstringspaces=false,
	numbers=left,
	numberstyle=\tiny,
	numberblanklines=true,
	showspaces=false,
	showtabs=false,
	xleftmargin=-10pt,
	xrightmargin=-20pt,
	emph=
	{[1]
		return,printf,fprintf,if,else,scanf,sscanf,sprintf,malloc,
		calloc,realloc,fgets,fputs,puts,system,strcmp,strstr,
		strchr,exit,for,while,fclose,fopen,atoi,sizeof,sin,cos
	},
	emphstyle={[1]\color{blue}\textbf},
	emph={[2]char,float,double,void,NULL,int,bool},
	emphstyle={[2]\color{green_ulises}\textbf},
	emph={[3]typedef, struct},
	emphstyle={[3]\color{castanho_ulisses}\textbf}
}

\lstdefinelanguage{txt}
{
       basicstyle=\ttfamily\scriptsize,
       showstringspaces=false,
       numbers=left,
       numberstyle=\tiny,
       numberblanklines=true,
       showspaces=false,
       showtabs=false
}
\lstnewenvironment{code_txt}{\lstset{language=txt}}{}
\lstnewenvironment{code_c}{\lstset{language=C_ulisses}}{}

\lstnewenvironment{code_lex}
{\textbf{Código Lex} \hspace{1cm} \hrulefill \lstset{language=txt}}
{\hrule\smallskip}

\title{\sf  Métricas para avaliação de Linguagens de Modelação - UML \\ Engenharia de Linguagens \\ Engenharia Gramatical \\
\begin{tabular}{c}
    \includegraphics[width=.1\textwidth]{stuff/uminho.jpg}
    \includegraphics[width=.07\textwidth]{stuff/informatica.jpg}\\
    {\small Universidade do Minho}, {\small MEI}\\
    {\small Ano lectivo 2010/2011}\\
    {\small Trabalho Prático}\\
\end{tabular}
}
\author{
    {\small José Pedro Silva - pg17628} \and
    {\small Mário Ulisses Costa - pg15817} \and
    {\small Pedro Faria - pg17684}}
\date{{\small \today}}


\begin{document}

\maketitle


\begin{abstract}
Foi pedido para a UCE de Engenharia de Linguagens, que fosse feito um estudo sobre dados disponíveis relativamente a 
métricas de Linguagens de Modelação, mais concretamente UML. No geral, a maior parte das métricas para UML foram adaptadas a partir
de métricas de Programação Orientada a Objectos, o que é o caso das CK metrics.
\end{abstract}


\section{CK Metrics}
As métricas para avaliação de  software estão bem mais desenvolvidas que as métricas para linguagens de modelação. É então interessante ter em conta o estudo já realizado sobre métricas para avaliação de código, como é o caso das CK Metrics, como ponto de partida para métricas de modelação.\\
As CK Metrics, umas das primeiras métricas para o modelo Orientado a Objectos (OO), foram propostas por Chidamber e  Kemerer (Chidamber et. al. 1991; Chidamber et. al. 1994). O conjunto das CK Metrics consiste em seis métricas: Weighted Methods Per Class (WMC), Depth of Inheritance Tree (DIT), Number of Children (NOC), Coupling between Object Classes (CBO), Response For a Class (RFC), e Lack of Cohesion in Methods (LCOM). Estas métricas foram depois adaptadas para linguagens de modelação. De seguida, será explicado como são calculadas cada uma das métricas referidas.

\section{Conclusão}

\section{Referências Bibliográficas}
http://www.pacis-net.org/file/2005/158.pdf
http://www.minds.nuim.ie/~jmcq/NUIM-CS-TR-2006-03.pdf
http://www.cs.nuim.ie/~jpower/Research/Papers/2006/modelsize06.pdf
\end{document}

