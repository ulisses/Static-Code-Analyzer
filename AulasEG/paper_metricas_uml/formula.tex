Temos então 4 grandes grupos de medidas que pudemos usar para definir as fórmulas que vão ser usadas para análise dos diagramas e consequentemente extracção de uma métrica
de qualidade para os diagramas UML.
\paragraph{Métricas primitivas} que consistem na extracção bruta de informaçao no que diz respeito á quantidade de métodos, classes, parametros, etc.
Temos que o número total de classes (\textit{Total Number of Classes}) é $TNC = \sum_{i=1}^{n} tnc_i $,
o número total de relações herdadas (\textit{Total Number of Inheritance Relationships})  $TNIR = \sum_{i=1}^{n} tnir_i $,
o número total de relações que ão sejam herdadas (\textit{ Total Number of Realization Relationships})  $TNRR = \sum_{i=1}^{n} tnrr_i $, por isto entende-se
uma relação entre dois elementos do modelo UML, entre dois diagramas de classes, cujo elemento cliente conhece o comportamento do outro elemento a que ele está ligado.
Temos ainda a contagem do número total de relações que existem (\textit{Total Number of Use Relationships})  $TNUR = \sum_{i=1}^{n} tnur_i $,
o número total de associações (\textit{Total Number of Associations})    $TNA = \sum_{i=1}^{n} tna_i $
(\textit{Total Number of Roles})   $TNR = \sum_{i=1}^{n} tnr_i $
(\textit{Total Number of Operation})   $TNO = \sum_{i=1}^{n} tno_i $
(\textit{Total Number of Parameters})  $TNP = \sum_{i=1}^{n} tnp_i $
(\textit{Total Number of Class Attributes})  $TNCA = \sum_{i=1}^{n} tnca_i $

\begin{description}
 \item [Fault-Proneness Metrics] - \begin{itemize}
                                   \item WMC - Weighted Method per Class  = $ \sum_{i=1}^{n} c_i $ , onde $c_i$ é a complexidade dos métodos.
				   \item NOC - Number of Children per Class = $ \sum_{i=1}^{n} sc_i $ , onde $sc_i$ é o número de subclasses imediatas.
				   \item DIT - Depth of Inheritance Tree  = $ max_leng $ , onde $ max_leng $ é o comprimento máximo desde a raiz até à folha.
                                  \end{itemize}
 \item [Quality Measure Metrics] - \begin{itemize}
                                   \item MHF - Method Hiding Factor = $ \frac{\sum_{i=1}^{rc} \sum_{m=1}^{Md(c_i)} (1-V(M_{mi}))} {\sum_{i=1}^{rc} Md(c_i)} $ onde: \begin{itemize}
																				      \item $V(M_{mi})$ = $\frac{\sum_{j=i}^{rc} is\_visible(M_{mi},C_j)}{TC-1} $
																				      \item $ is\_visible(M_{mi},C_j) $ = \( \left \{ \begin{array}{l l}
																									      1 & \quad iff \left \{ \begin{array}{l l}
																									                              j \neq i \\
																												      C_j \texttt{may call} M_{mi}\\
																									                             \end{array}
\right.\\
																									      0 & \quad otherwise\\
																									    \end{array} \right. \)
																				      \item TC $=$ Número total de classes
																				      \item Md $=$ Número total de métodos definidos
																				      \item $V(M_{mi})$ $=$ A visibilidade de todas as classes onde o método $M_{mi}$ é visível
																				      \item MHF é então a medida do uso de informação através de métodos
																				    \end{itemize}
				   \item AHF - Attribute Hiding Factor = $ \frac{\sum_{i=1}^{rc} \sum_{m=1}^{Ad(c_i)} (1-V(A_{mi}))} {\sum_{i=1}^{rc} Ad(c_i)} $ onde: \begin{itemize}
																				      \item $V(A_{mi})$ = $\frac{\sum_{j=i}^{rc} is\_visible(A_{mi},C_j)}{TC-1} $
																				      \item $ is\_visible(A_{mi},C_j) $ = \( \left \{ \begin{array}{l l}
																									      1 & \quad iff \left \{ \begin{array}{l l}
																									                              j \neq i \\
																												      C_j \texttt{may call} A_{mi}\\
																									                             \end{array}
\right.\\
																									      0 & \quad otherwise\\
																									    \end{array} \right. \)
																				      \item TC $=$ Número total de classes
																				      \item Ad $=$ Número total de atributos definidos
																				      \item $V(A_{mi})$ $=$ A visibilidade de todas as classes onde o atributo $A_{mi}$ é visível
																				      \item AHF é então a medida do uso de informação através de atributos
																				    \end{itemize}
				   \item MIF - Method Inheritance Factor = $ \frac{\sum_{i=1}^{rc} M_i(C_i)}{\sum_{i=1}^{rc} M_a(C_i)} $ onde: \begin{itemize}
				                                                                                                                \item $ M_a(C_i) = Md(C_i) + M_i(C_i)$, é o número total de métodos disponíveis(definidos localmente mais os herdados)
				                                                                                                                \item MIF é então a medida de herdagem através de métodos
				                                                                                                               \end{itemize}
				   \item AIF - Attribute Inheritance Factor = $ \frac{\sum_{i=1}^{rc} A_i(C_i)}{\sum_{i=1}^{rc} A_a(C_i)} $ onde: \begin{itemize}
				                                                                                                                \item $ A_a(C_i) = Ad(C_i) + A_i(C_i)$, é o número total de atributos disponíveis(definidos localmente mais os herdados)
				                                                                                                                \item MIF é então a medida de herdagem através de atributos
				                                                                                                               \end{itemize}

                                  \end{itemize}
 \item [Use Case Metrics] - Para se obter complexidade dinâmica através da informação obtida de um diagrama Use Case:\begin{itemize}
                             \item NOA - Number of actors = $ \sum_{i=1}^{n} noa_i $
                             \item NOUC - Number of use cases = $ \sum_{i=1}^{n} nouc_i $
                             \item NOUCA - Use cases per Actor = $ \sum_{i=1}^{n} nouca_i $
                            \end{itemize}
  
\end{description}
