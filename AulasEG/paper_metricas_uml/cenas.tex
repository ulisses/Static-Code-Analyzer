--estas metricas sao importantes para model driven arquitecture ?\\

O conjunto de métricas do capítulo anterior são referidos em vários papers e sem sombra de dúvidas são as mais estudadas, e também as mais utilizadas para avaliar modelos UML.\\
Estas focam-se mais nos diagramas de classes visto estes serem os que mais facilmente se relacionam com código e é preciso ter em consideração que como estas regras derivam 
directamente do paradigma Orientado-a-Objectos, é mais fácil aplicá-las aos diagramas de classes. Para além disso este tipo de diagrams do ponto de vista da implementação 
dão uma visão mais geral do sistema que modela.\\

Estas métricas em particular são detalhadas por McQuillan e Power em~\cite{Power}. Existe também um software~\cite{SDMetrics}(SDMetrics) que avalia além destas, um conjunto
 mais extenso de métricas, que analisa outros diagramas além do de classes, como por exemplo os diagramas de estados e de actividades. \\ 
Como grande parte das métricas derivam de fórmulas matemáticas, no capítulo seguinte serão apresentadas algumas que são essenciais para o cálculo dos valores das métricas apresentadas 
anteriormente.

