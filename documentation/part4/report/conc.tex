\chapter{Conclusão e Trabalho Futuro}\label{chap con} 
Ao longo de toda primeira fase modelamos os vários aspectos do nosso sistema. Na segunda fase partimos para a implementação do
sistema e focamos também a nossa atenção na exploração de um frontend para a linguagem C. Na terceira fase deu-se continuidade à  implementação de uma interface em linha de comandos que fizesse algumas das tarefas que se podem fazer através da interface Web. Também se aprofundou conhecimentos sobre o frontend \textit{Language.C}.\\
\\
Toda a modelação do sistema realizada, foi importante, devido à visão mais alargada que nos deu do problema e da sua resolução.
A implementação correu de forma tranquila, onde apenas pequenos ajustes se realizaram, em relação ao pensado primeira fase.
O trabalho relativo à importação de enunciados e tentativas em formato XML, que tinha ficado incompleto na fase I, foi na fase II
completado.\\
\\
Na fase II foi também dado inicio ao desenvolvimento de uma interface em linha de comandos escrita em Perl, que alarga a acessibilidade à
aplicação, deixando do acesso à mesma estar dependente de um browser.\\
\\
Ainda nesta fase a exploração do frontend \textit{Language.C} também deu os primeiros passos.\\
No fim da terceira fase, o frontend escolhido foi completamente dominado e partiu-se para a exploração de técnicas de programação genérica que
implementasse estratégias de percorrer árvores de parsing, para que se conseguisse extraír informação sobre o código que recebemos de input.\\
\\
O sistema, agora na última fase, está apto a receber as soluções dos utilizadores, tratá-las, apresentar resultados e fazer detecção de clonning. Preparou-se também o sistema para gerar e permitir o download de um relatório em pdf sobre os resultados obtidos nas tentativas submetidas pelos utilizadores, e também um relatório sobre as métricas extraídas pelo analisador criado. A aplicação Web ficou assim terminada.\\
O analisador de métricas desenvolvido implementou todas as métricas pretendidas. Além disso foi criada uma API de metricas e um detector de clones. Este analisador gera um relatório em pdf, que além das métricas também apresenta os possíveis clones encontrados.\\
O output do analisador deixa assim ao dispor do utizador tudo o que precisas para extrair uma noção de qualidade do software submetido.

