\newcommand{\metricsins}{\mathbin{>\mkern-7mu\circ\mkern-9mu>}}
\newcommand{\metricscat}{\mathbin{>\mkern-7mu+\mkern-11mu>}}
\newcommand{\bind}{\mathbin{>\mkern-7mu>\mkern-4mu=}}

\section{Metrics and Assessment}\label{metrics}
\subsection{API design}
To be able to manipulate and store the calculated metrics we have develop an API. Here are the relevent data types:
\begin{haskell*}
  \hskwd{type} Metrics
    & = & Map MetricName MetricValue\\
  \hskwd{type} MetricName
    & = & (Name, Maybe FileName, Maybe FunctionName)\\
  \hskwd{data} MetricValue
    & = & Num Double\\
    & | & Clone (Map FileDst [(Ocurrency, LineSrc, LineDst)])\\
    & | & Includes ([SystemIncludes],[Includes])
\end{haskell*}

\noindent We have used a Map to store our metrics, where the key is the triple $MetricName$. Some metrics just make sense regarding all the software package,
others may only be relevant regarding files and other may be extracted from a function. By letting some of the $Maybe$ fields qith the value $Nothing$ we express
this three categories of metrics.
Regarding the $MetricValue$ we can have multiple types of metrics, most of them just return a numerary value, the cosntructor $Num$. And then we have more specific
metrics related to clones found in the code, list of include and system includes.\\
\indent Now we present one of the most important function to append metrics. The power of \textit{Haskell} infix notation allow us to use comfortably this kind of functions.

\begin{haskell*}
  \metricsins ::  Metrics  &\to& (MetricName, MetricValue) \to Metrics\\
  m \metricsins (mn,mv) &=& \hscase{lookup mn mv}{
        \hskwd{Nothing} &\to& m'\\
        (\hskwd{Just} mv') &\to& \hskwd{if} mv' \equiv mv \hskwd{then} m \hskwd{else} m'
	}
	\hswhere{m' = insert mn mv m}
\end{haskell*}

The previous function can be used when we want create a new metric, if so we must use $emptyMetrics$ as first paramenter.
\indent The following function can be used when the user already have some metrics calculated and want to concat them into one.

\begin{haskell*}
\metricscat & :: & Metrics \to Metrics \to Metrics\\
m_1 \metricscat m_2 & = & union m_1 m_2
\end{haskell*}

As an example we can use this function to take a list of $Metrics$ and concat all of them into just one $Metrics$ value. If we have a function
$$mccabeIndex~::~FileContents \to IO~Metrics$$
We can apply this function to a list of files $lst$ and concat the results into one single $Metrics$ bag:
$$mapM~mccabeIndex~lst\bind return \circ foldl~(\metricscat)~emptyMetrics$$

\subsection{Metrics}
Our system is able to calculate some well known metrics. We have divided this metrics in some groups: Metrics over comments, Complexity metrics, Metrics over functions,
Metrics over includes and Metrics over lines quantity.
\paragraph{Comments metrics} are used to understand the overall matureness of the software project, so we consider \textit{lines of comments (lc)} as the non code comments
number of lines and \textit{comment lines density} as $cld = \frac{lc}{nrLines + lc} * 100$.
\paragraph{Complexity metrics} for now our system is only capable of mesure one metric, Cyclomatic Complexity from one overall file or from a function. This
metric is calculates giving the parsing tree of a C file (or C function) and for everyone condition in the code increment the value $1$.
So, for each $If$, $Switch$, $For$ and $While$ statement we add $1$ to our result.
\paragraph{Functions metrics and Includes metrics} are not exactly quality metrics, but more a global view of the software project. We provide the following metrics:
\textit{get functions name} that returns a list of functions names from a parsing tree and \textit{get functions signature} that returns a list of a signature node in the
parsing tree (we can use pretty print functions provided by Language.C to print them out). For Includes metrics, we provide a function that receives the file name and
returns a tuple containing two lists: the system includes and the library includes.
\paragraph{Metrics over lines} where we extract some metrics related to lines, so for this metrics we do not use Language.C library.
We implement \textit{number of lines of code (ncloc)}, \textit{number of physical lines} and two functions to exctract line clones and block of lines clones (one block
could be 3 lines).


