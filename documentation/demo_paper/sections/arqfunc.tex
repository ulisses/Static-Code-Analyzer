\section{Architecture and features}
As said before, our tool is a system for managing programming contests. It can be accessed trough a Web Interface which is a \textit{RoR}\footnote{Ruby on Rails - \url{http://rubyonrails.org/}} application, and some of it's features can also be used from the terminal, taking advantage of a \textit{Perl} written script.
In Figure \ref{fig:arq}, we present a diagram that shows a simplified version of the system architecture.

\begin{figure}[htbp]
\includegraphics[scale=0.7]{images/arq}
\caption{Simplified System Architecture}
\label{fig:arq}
\end{figure}

\subsection{\textit{RoR} interface and \textit{Perl} interface}
As it would be expected, our system allows the creation of new contests, where among other things,
we can stipulate the date in which the contest starts and ends, and also it's durations.
Several exercises can be added to each contest, either manually, or by submitting one or more xml files.
An exercise has a description, a set of languages in which the competitors can solve the exercice, and a set of
input's and output's that the program will try to pass.
Having the contest propely created and being available to the competitors, they can register to be a contest participant,
and then start submitting their solutions to each of the contest exercises.  

\subsection{Parser and Traversal Strategies}
All of the work related to C files parsing, metrics extraction from them and output generation is done
using Haskell programming language. To extract some metrics we use a parser
library\footnote{Language.C - http://trac.sivity.net/language\_c/} done by Benedikt Huber under a GSoC (Google summer of code).
With this library we are able to construct a small parsing tree with the complete AST of C~\cite{Kernighan:1988:CPL:576122} and some GNU C extensions.\\
\indent To have an idea we only have 84 types of nodes (type constructors) in this tree, even so it is too big to traverse it by constructor, so
it was mandatory to use strategies to traverse this tree. We decide to use Strafusnki library~\cite{LV03-PADL}.

