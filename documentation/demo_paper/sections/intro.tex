\section{Introduction}
Quantitative measurements are essential in all sciences, and computer science is no exception.
Although Software Metrics aren't often used in Software Development, there has been an effort on the computer science community to develop and improve this metrics. 
The goal is to make them valuable in many aspects, such as quality assurance.\\
We present a static code analyser, that by measuring a set of metrics over C code, will give a notion of quality to the user. The software is able to generate
reports in \LaTeX~and in \textit{XML} format.\\
A system for managing programming contests, was chosen as the case study.
The competitors submit source code, which is their attempt to resolve a given problem.
The goal is not only to check if the solution is correct, but evaluate the quality of the submited code.\\
%We proceed to the next section, where we will show the system architecture, it's features and explain what is a programming contest, in this context.
%Ending with the explanation of how the parsing tree for the metrics calculation is constructed.
%We also explain the implementation of the case study, which is available in two differente plataforms.
%In the last section, we introduce the \textit{API} of the Metrics calculation, and a summarized explanation of each calculated metric.

