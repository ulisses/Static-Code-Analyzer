\chapter{Conclusão e Trabalho Futuro}\label{chap con} 
Ao longo de toda primeira fase modelamos os vários aspectos do nosso sistema. Na segunda fase partimos para a implementação do
sistema e focamos também a nossa atenção na exploração de um frontend para a linguagem C.\\
\\
Toda a modelação do sistema realizada, foi importante, devido à visão mais alargada que nos deu do problema e da sua resolução.
A implementação correu de forma tranquila, onde apenas pequenos ajustes se realizaram, em relação ao pensado na fase anterior.
O trabalho relativo à importação de enunciados e tentativas em formato XML, que tinha ficado incompleto na fase I, e foi na fase II
completado.\\
O sistema está no fim desta fase apto a receber as soluções dos utilizadores, tratá-las, apresentar resultados e fazer detecção de clonning.\\
Na fase II foi também dado inicio ao desenvolvimento de uma interface em linha de comandos escrita em Perl, que alarga a acessibilidade à
aplicação, deixando do acesso à mesma estar dependente de um browser.\\
Ainda nesta fase a exploração do frontend \textit{Language.C} também deu os primeiros passos.\\
Neste momento da fase III esta ferramenta esta completamente dominada e partimos para a exploração de técnicas de programação genérica que
implemente estratégias de percorrer árvores de parsing, para conseguirmos extraír informação sobre o código que recebemos de input.
\\
Cremos que no fim desta fase III, o desenvolvimento da aplicação está terminado, foram corrigidos alguns bugs, a
interface está mais apelativa e intuitiva.\\

Para a próxima fase espera-nos o término da a aplicação pelo terminal e a aplicação de muito mais métricas e geração de um report com os resultados.
