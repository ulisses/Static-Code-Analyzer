%%!TEX encoding = UTF-8 Unicode 
\documentclass[english,a4paper]{report}
%%%%%%%%%%%%%%%%%%%%%%%%%%%%%%%%%%%%%%%%%%%%%%%%%%%%%%%%%%%%%%%%%%%%%%%%%%%%%%%%%%%%%%%%%%%%%%%%%%%%%%%%%%%%%%%%%%
\usepackage[portuges]{babel}
\usepackage[portuges]{minitoc}
\usepackage[utf8]{inputenc}
\usepackage[T1]{fontenc}
\usepackage[pdftex]{color,graphicx}
% pretty collors.. :)
\usepackage[a4paper, pdftex, bookmarks, colorlinks,citecolor=darkblue,linkcolor=darkblue,urlcolor=darkblue,filecolor=darkblue]{hyperref}
%\usepackage{t1enc}
%\usepackage{aeguill}
 \usepackage[avantgarde]{quotchap}
\usepackage{varioref}
\usepackage{psboxit}
\usepackage{fancybox}
\usepackage{fancyvrb}
\usepackage{color}
\usepackage{cite}
\usepackage{hyperref}
\usepackage{graphicx}
\usepackage{float}
\usepackage{listings}
\usepackage{geometry}
\usepackage{graphics}
\usepackage{setspace}
\usepackage{thumbpdf}
\usepackage{fancyhdr}
\usepackage{lastpage}
\usepackage{alltt}
\usepackage{graphicx}
\usepackage{verbatim}
\usepackage{url}
\usepackage{epsf}
\usepackage{listings}
\usepackage[refpages]{gloss}
\usepackage{array}
\usepackage{longtable}
\usepackage{multirow}
\usepackage{amsmath, amsthm, amssymb}
\usepackage{slashbox}
\usepackage{rotating}
%%%%%%%%%%%%%%%%%%%%%%%%%%%%%%%%%%%%%%%%%%%%%%%%%%%%%%%%%%%%%%%%%%%%%%%%%%%%%%%%%%%%%%%%%%%%%%%%%%%%%%%%%%%%%%%%%%

\definecolor{darkblue}{rgb}{0,0.1,0.5}
%\definecolor{darkred}{rgb}{0.8,0,0}
% \def\tagform@#1{\maketag@@@{\cornersize{0.8}\ovalbox{\ignorespaces\sffamily{#1}\unskip\@@italiccorr}}}
%% first reset the headers and footers
\fancyhead{}
\fancyfoot{}
%% make the odd pages have the section name on the top right
\fancyhead[RO]{\sffamily\bfseries \rightmark}
%% make the even pages have the chapter name on the top left
\fancyhead[LE]{\sffamily\bfseries \leftmark}

%% page nums on the bottom in a nice box
%% even side pages
\fancyfoot[LE]{\psboxit{box 0.8 setgray fill}
{\framebox[10mm][c]{\rule{0cm}{4mm}\color{black}{\bfseries \thepage}}}}
%% odd side pages
\fancyfoot[RO]{\psboxit{box 1 setgray fill}
{\hspace{\textwidth}\psboxit{box 0.8 setgray fill}
{\framebox[10mm][c]{\rule{0cm}{4mm}\color{black}{\bfseries \thepage}}}}}

%% make the bottom line above the page number box
\renewcommand{\footrulewidth}{0.4pt}
\renewcommand{\footruleskip}{0mm}

\pdfpagewidth=\paperwidth
\pdfpageheight=\paperheight

\renewcommand\familydefault{\sfdefault}% usar font sem serifas

% definir acrónimos com itálico
%\renewcommand*{\acf}[1]{\acffont{\textit{\acl{#1}}~\acfsfont{(\acs{#1})}}}

\newtheorem{defin}{Definição}

\pagestyle{fancy}
%\lhead{}
%\rhead{}

%% now redefine the plain style pages (chapter pages, contents pages)
%% to have the same page number stuff on the bottom
\fancypagestyle{plain}{
	\fancyhf{}
	\fancyfoot[RO]{\psboxit{box 1 setgray fill}
	{\hspace{\textwidth}\psboxit{box 0.8 setgray fill}
	{\framebox[10mm][c]{\rule{0cm}{4mm}\color{black}{\bfseries \thepage}}}}}
	\renewcommand{\headrulewidth}{0pt}
	\renewcommand{\footrulewidth}{0.5pt}
}
% %%%%%%%%%%%%%%%%%%%%%%%%%%%%%%%%%%%%%%%%%%%%%%%%%%%%%%%%%%%%%%%%%%
\definecolor{gray_ulisses}{gray}{0.55}
\definecolor{castanho_ulisses}{rgb}{0.71,0.33,0.14}
\definecolor{preto_ulisses}{rgb}{0.41,0.20,0.04}
\definecolor{green_ulises}{rgb}{0.2,0.75,0}

%% stuff do minitoc %%%%%%%%%%%%%%%%%%%%%%%%%%%%%%%%%%%%%%%
\setcounter{minitocdepth}{2}
\setlength{\mtcindent}{24pt}
\renewcommand{\mtcfont}{\small\rm}
\renewcommand{\mtcSSfont}{\small\bf}
%\newenvironment{mtc}{\secttoc\sectlof\sectlot}{\pagebreak}
%                        ^       ^        ^
%                    conteudos  figuras  tabelas
% \newenvironment{mtc}{\minitoc\minilof\minilot}{\pagebreak}
%%%%%%%%%%%%%%%%%%%%%%%%%%%%%%%%%%%%%%%%%%%%%%%%%%%%%%%%%%%
\lstdefinelanguage{HaskellUlisses}
{
        basicstyle=\ttfamily\scriptsize,
        %backgroundcolor=\color{yellow},
        %frameshape={RYRYNYYYY}{yny}{yny}{RYRYNYYYY}, %contornos... muito nice...
        sensitive=true,
        morecomment=[l][\color{gray_ulisses}\ttfamily\tiny]{--},
        morecomment=[s][\color{gray_ulisses}\ttfamily\tiny]{\{-}{-\}},
        morestring=[b]",
        stringstyle=\color{red},
        showstringspaces=false,
%       numbers=left,
%       firstnumber=\thelstnumber,
        numberstyle=\tiny,
        numberblanklines=true,
        showspaces=false,
        breaklines=true,
        showtabs=false,
%       xleftmargin=15pt,
%       xrightmargin=-20pt,
        emph=
        {[1]
                FilePath,IOError,abs,acos,acosh,all,and,any,appendFile,approxRational,asTypeOf,asin,
                asinh,atan,atan2,atanh,basicIORun,break,catch,ceiling,chr,compare,concat,concatMap,
                const,cos,cosh,curry,cycle,decodeFloat,denominator,digitToInt,div,divMod,drop,
                dropWhile,either,elem,encodeFloat,enumFrom,enumFromThen,enumFromThenTo,enumFromTo,
                error,even,exp,exponent,fail,filter,flip,floatDigits,floatRadix,floatRange,floor,
                fmap,foldl,foldl1,foldr,foldr1,fromDouble,fromEnum,fromInt,fromInteger,fromIntegral,
                fromRational,fst,gcd,getChar,getContents,getLine,head,id,inRange,index,init,intToDigit,
                interact,ioError,isAlpha,isAlphaNum,isAscii,isControl,isDenormalized,isDigit,isHexDigit,
                isIEEE,isInfinite,isLower,isNaN,isNegativeZero,isOctDigit,isPrint,isSpace,isUpper,iterate,
                last,lcm,length,lex,lexDigits,lexLitChar,lines,log,logBase,lookup,map,mapM,mapM_,max,
                maxBound,maximum,maybe,min,minBound,minimum,mod,negate,not,notElem,null,numerator,odd,
                or,ord,otherwise,pi,pred,primExitWith,print,product,properFraction,putChar,putStr,putStrLn,quot,
                quotRem,range,rangeSize,read,readDec,readFile,readFloat,readHex,readIO,readInt,readList,readLitChar,
                readLn,readOct,readParen,readSigned,reads,readsPrec,realToFrac,recip,rem,repeat,replicate,return,
                reverse,round,scaleFloat,scanl,scanl1,scanr,scanr1,seq,sequence,sequence_,show,showChar,showInt,
                showList,showLitChar,showParen,showSigned,showString,shows,showsPrec,significand,signum,sin,
                sinh,snd,span,splitAt,sqrt,subtract,succ,sum,tail,take,takeWhile,tan,tanh,threadToIOResult,toEnum,
                toInt,toInteger,toLower,toRational,toUpper,truncate,uncurry,undefined,unlines,until,unwords,unzip,
                unzip3,userError,words,writeFile,zip,zip3,zipWith,zipWith3,listArray,doParse
        },
        emphstyle={[1]\color{blue}},
        emph=
        {[2]
                Bool,Char,Double,Either,Float,IO,Integer,Int,Maybe,Ordering,Rational,Ratio,ReadS,ShowS,String,
                Word8,InPacket
        },
        emphstyle={[2]\color{castanho_ulisses}},
        emph=
        {[3]
                case,class,data,deriving,do,else,if,import,in,infixl,infixr,instance,let,
                module,of,primitive,then,type,where
        },
        emphstyle={[3]\color{preto_ulisses}\textbf},
        emph=
        {[4]
                quot,rem,div,mod,elem,notElem,seq
        },
        emphstyle={[4]\color{castanho_ulisses}\textbf},
        emph=
        {[5]
                EQ,False,GT,Just,LT,Left,Nothing,Right,True,Show,Eq,Ord,Num
        },
        emphstyle={[5]\color{preto_ulisses}\textbf}
}
\lstnewenvironment{haskell}{\lstset{language=HaskellUlisses}}{}

\lstdefinelanguage{files} {
       basicstyle=\ttfamily\scriptsize,
       showstringspaces=false,
       showspaces=false,
       showtabs=false
}
\lstnewenvironment{code_files}{\lstset{language=files}}{}

\lstdefinelanguage{xxml} {
    basicstyle=\ttfamily\scriptsize,
	numbers=left,
	numberstyle=\tiny,
	numbersep=5pt,
	breaklines=true,
	mathescape=true,
	frame=tB
}
\lstnewenvironment{myxml}{\lstset{language=xxml}}{}

\geometry{verbose,a4paper,tmargin=30mm,bmargin=30mm,lmargin=30mm,rmargin=30mm}

%%%%%%%%%%%%%%%%%%%%%%%%%%%%%%%%%%%%%%%%%%%%%%%%%%%%%%%%%%%%%%%%%%%%%%%%%%%%%%%%%%%%%%%%%%%%%%%%%%%%%%%%%%%%%%%%%%
\begin{document}
\begin{titlepage}
\thispagestyle{empty}
\begin{figure}[htbp]
\begin{center}
\includegraphics[width=0.3\textwidth]{Images/DI-UM}
\end{center}
\end{figure}
{\centering \large
{\large\bf \textbf{Universidade do Minho} \\ Departamento de Informática}\\
\vspace{1cm}
\bf{Engenharia de Linguagens}\\
\vspace{2cm}
{\Large \bf {Projecto Integrado}}\\
\vspace{1cm}
{\LARGE \bf {\emph{Software} para Análise e Avaliação de Programas}}\\
\vspace{1cm}
%{\large \bf {}}
\vspace{8.5cm}
}

\flushleft{ \emph{Grupo 2:\\}
\vspace{0.4cm}
\begin{tabular}{ll}
José Pedro Silva & pg17628 \\
Mário Ulisses Costa & pg15817 \\
Pedro Faria & pg17684 \\
\end{tabular}
\\
}
\vspace {1.5cm}
\textbf{Braga, \today}\\
\pagebreak
\end{titlepage}

\pagenumbering{roman}
\thispagestyle{plain}
\chapter*{Abstract}

Este relatório trata da primeira fase do Projecto Integrado (PI) da UCE de Engenharia de Linguagens. O PI consiste num sistema
de sbmissão de trabalhos dos alunos. Este sistema deve ser capaz de aceitar registos de alunos e professores. Deve ainda permitir a submissão
de código por parte dos alunos e a submissão de exercicios por parte dos professores. O sistema deve ainda ser capaz de fazer uma análise detalhada
sobre o código submetido pelo aluno e gerar um relatório para o professor ler sobre este mesmo código.\\

Nesta fase o grande objectivo é modelar todo este sistema, decidimos fazê-lo de um ponto de mais técnico, com alguns diagramas e estrutura dos ficheiros XML que vão
ser úteis para a nossa implementação, mas também decidiu-se fazer uma modelação do ponto de vista mais formal, de forma a nesta fase clarificar o processo de
modelação e fazer com que dêmos maior foco à modelação em si. Semt er de pensar nos sistemas complexos qe estão por trás disso.


% indices
\dominitoc
\dominilof
\dominilot
\renewcommand{\contentsname}{Índice}
\tableofcontents
\addcontentsline{toc}{chapter}{\contentsname}
\renewcommand{\listfigurename}{Índice de Figuras}
\listoffigures
\addcontentsline{toc}{section}{\listfigurename}
\renewcommand{\listtablename}{Índice de Tabelas}
\listoftables
\addcontentsline{toc}{section}{\listtablename}
 

\newpage
\pagenumbering{arabic}
\addtocounter{mtc}{+2}
\newpage
\chapter{Introdução} \label{chap int}
Este relatório descreve o projecto desenvolvido para o módulo Projecto Integrado da UCE de Engenharia Linguagens do Mestrado em Engenharia Informática da Universidade do Minho.\\

Pretende-se que este projecto seja um \emph{Software} para Análise e Avaliação de Programas (SAAP), tendo este \emph{software} como objectivo criar um 
ambiente de trabalho, com um interface Web, que permita a docentes/alunos avaliar/submeter programas automaticamente.

\section{Motivação}
Este trabalho é muito interessante do ponto de vista de produto final, como obra didáctica para a sua concretização. Para a sua leitura os autores terão de aplicar
conhecimentos adquiridos na área de arquitectura de um sistema de informação, desenvolvimento para a web, linguagens de scripting, bases de dados, processamento de textos, etc\ldots\\
Como se pode facilmente concluir, este é um projecto que do ponto de vista técnico é muito motivador devido à sua magnitude e inovação técnica que implica aos seus autores.

\section{Objectivos}
Este projecto tem como objectivos consolidar conhecimentos adquiridos nos diferentes módulos da UCE de Engenharia de Linguagens
Assim sendo, vamos recorrer a tecnologia aprendida durante as aulas, bem como tecnologia que já conheciamos e aprendemos durante a nossa formação académica ou à parte desta.
Um dos nossos grandes objectivos pessoais é consolidar e desenvolver ainda mais o uso de tecnologias variadas, para arranjar uma solução elegante, funcional e que cumpra os requesitos do sistema descrito.
Assim, este projecto é mais do que um projecto de mestrado, passando a ser encarado como um desafio ao conhecimento e a pôr em prática conhecimentos adquiridos.

\subsection{Ferramentas}
As ferramentas/tecnologias que pensamos utilizar são as seguinte:
\begin{itemize}
\item DB2
\item Perl
\item Ruby on Rails (RoR)
\item Haskell
\end{itemize}

Iremos usar DB2 por suportar nativamente XML e ainda XPath e XQuery para fazer travessias no XML. O uso de Perl prende-se com o facto de ser incrivelmente fácil criar
sem muito esforço pequenas ferramentas que acreditamos serem uteis para identificar padrões em texto ou cortar pequenos pedaços de texto.\\
O uso de RoR deve-se ao facto da simplicidade em criar ambientes web. Decidimos ainda utilizar Haskell eventualmente em tarefas mais complexas, por ser uma linguagem de rápida
implementação e bastante segura.


\section{Descrição do Sistema}
O SAAP - Software para Análise e Avaliação de Programas é um sistema disponível através de uma interface web, que terá como principal função 
submeter, analisar e avaliar automaticamente programas.\\
O sistema estará disponível em parte para utilizadores não registados, mas a suas principais funcionalidades estarão apenas disponíveis para docentes 
e grupos já registados no sistema.\\
O sistema poderá ser utilizado para várias finalidades, no entanto estará direccionado para ser utilizado em concursos de programação e em elementos 
de avaliação universitários.

\subsection{Utilizadores}
Existem três entidades que podem aceder ao sistema.\\ \\
O administrador, que além de poder aceder às mesmas funcionalidades do docente, funcionalidades essas que já iremos descrever, é quem tem o poder de
criar contas para os docentes.\\
\\
O docente tem acesso a todo o tipo de funcionalidades relacionadas com a criação, edição e eliminação de concursos e enunciados, assim como consulta
de resultados dos concursos e geração/consulta de métricas para os ficheiros submetidos no sistema.\\
\\
O grupo, que pode ser constituído por um ou mais concorrentes, terá acesso aos concursos disponíveis, poderá tentar registar-se nos mesmos, e submeter
tentativas de resposta para cada um dos seus enunciados.\\
\\
Além do que já foi referido, todos os utilizadores podem editar os dados da sua conta.\\ 
Falta referir que um utilizador não registado (guest), pode criar uma conta para o seu grupo, de modo a poder entrar no sistema.

\subsection{Funcionalidades do sistema}
As várias funcionalidades do sistema já foram praticamente todas mencionadas, vamos no entanto tentar explicar a sua maioria, com um maior nível de
detalhe.

\subsubsection{Criação de contas de grupo e de docente}
\begin{description}
 \item[Criação de conta de grupo:] Qualquer utilizador não registado poderá criar uma conta no sistema para o seu grupo, através da página principal do sistema.
Terá de preencher dados referentes ao grupo e aos respectivos concorrentes.
 \item[Criação de conta de docente:] Como já foi referido, o administrador terá acesso a uma página onde poderá criar contas para docentes.
\end{description}

\subsubsection{Criação de concursos e enunciados}
Tanto o administrador como o docente podem criar concursos. Depois de preencher todos os dados do concurso podem passar à criação de enunciados.
Nesta altura caso prefiram, podem importar enunciados previamente criados em xml.
O concurso só ficará disponível na data definida aquando da criação do concurso.

\subsubsection{Registo e participação nos concursos}
Um grupo que esteja autenticado no sistema pode tentar registar-se num dos concursos disponíveis, e será bem sucedido se a chave que utilizar for a correcta.\\
Depois de se registar no concurso, o tempo para a participação no mesmo inicia a contagem decrescente e o grupo poderá começar a submeter tentativas para
cada um dos enunciados. O sistema informará, pouco depois da submissão, se a resposta estaria correcta ou não.
Depois do tempo esgotar o grupo não pode submeter mais tentativas.

\subsubsection{Geração das métricas para os ficheiros submetidos}
O docente ou o administrador a qualquer altura podem pedir ao sistema que gere as métricas para as tentativas submetidas.

\subsubsection{Consulta dos ficheiros com as métricas e dos resultados do concurso}
O docente pode aceder aos logs que contêm a informação sobre as tentativas submetidas, ou se desejar, apenas aceder a informações mais específicas, tal como 
qual exercício tentou submeter determinado grupo, e se teve sucesso ou não.\\
Pode também visualizar os ficheiros com as informações das métricas.

\section{Estrutura do Relatório}
Este documento foi pensado em duas sec. São elas:\\

\textbf{\textit{Capítulo 1º}} O leitor é introduzido ao tema abordado, e é justificada a utilidade do projecto.\\

\textbf{\textit{Capítulo 2º}} Neste capítulo é explicada a modelação e todas as decições tomadas para a modelação do sistema a que nos propomos a concretizar.\\

\textbf{\textit{Capítulo 3º}} No capítulo final são tecidos alguns comentários relativos ao trabalho efectuado e motivação para trabalho futuro.\\

\newpage
\part{Milestone I}
\newcommand{\rarrow}{\rightarrow}
\newcommand{\larrow}{\leftarrow}
\newcommand{\unif}{\sim}
\def\prop#1#2#3{\noindent\\$\begin{array}{l} \{#1\} \\ #2 \\ \{#3\} \\ \end{array}$\\\\}

\chapter{Modelação do Problema} \label{chap modprob}

Contextualização do cap 3

\section{Modelação Informal}\label{sec modinf}
Com o diagrama da arquitectura do sistema, figura~\ref{fig diaact},pretende-se mostrar as várias entidades que podem aceder ao sistema, assim como as várias
actividades que cada uma pode realizar e tarefas para o sistema processar.
Também é realçada a ideia de que alguns dos recursos do sistema só estão dispóniveis ao utilizador depois 
de passar por outros passos, ou seja, o diagrama dá a entender a ordem pelas quais o utilizador e o sistema podem/devem executar as tarefas.\\

\begin{figure}[htbp]
\begin{center}
\includegraphics[width=0.9\textwidth]{Images/EL-PI}
\caption{Arquitectura do sistema}\label{fig diaact}
\end{center}
\end{figure}

Para começar, como temos duas entidades diferentes que podem aceder ao sistema (o docente e o aluno/concorrente), 
dividiu-se o diagrama em duas partes distintas (uma para cada entidade referida), de modo a facilitar a leitura.\\

Em ambos os casos, o login é a primeira actividade que pode ser realizada.
Se o login não foi efectuado com sucesso, é adicionado no log file uma entrada com a descrição do erro.
No caso de o login ser efectuado com sucesso, consoante as permissões do utilizador em questão, tem diferentes opções ao seu dispôr.\\

No caso do login pertencer a um docente, este terá acesso aos dados de cada um dos grupos, podendo verificar os resultados que estes 
obtiveram na resolução das questões do(s) concurso(s), assim como ao ficheiro que contém a análise das métricas dos vários programas submetidos 
pelos mesmos.
Poderá também criar novos concursos e os seus respectivos exercícios, assim como adicionar baterias de testes para os novos exercícios, 
ou para exercícios já existentes.\\

No caso do login pertencer a um aluno/concorrente, o utilizador terá a opção de se registar num concurso ou de seleccionar um no qual já 
esteja registado.
Já depois de seleccionar o concurso, pode ainda escolher o exercício que pretende submeter.
Depois de submeter o código fonte do programa correspondente ao exercício escolhido, e já sem a interacção do utilizador, 
o sistema compilará e tentará executar os diferentes inputs da bateria de testes do exercício, e compararar os resultados obtidos com os esperados.
No fim de cada um destes procedimentos, serão guardados os resultados / erros.
Para terminar, será feito um estudo das métricas do ficheiro submetido, tendo como resultado a criação um ficheiro com os dados relativos a essa avaliação.\\

\section{Modelação Formal}\label{sec modfor}

Afim de haver algum rigor na definição do sistema decidiu-se fazer um modelo mais formal do ponto de vista dos dados e das funcionalidades que o sistema apresenta.
A ideia desta modelação é ainda ser um modelo formal da especificação atrás descrita. Para isso utilizou-se uma notação orientada aos contratos (design by contract),
com a riqueza que as pré e pós condições de funções nos oferecem.\\

Assim sendo, temos descritos os contratos da seguinte forma:
$$\noindent\begin{array}{c} \{P\} \\ C \\ \{R\} \\ \end{array}$$
em que $P$ define uma pré condição, $C$ uma assinatura de uma função e $R$ uma pós condição.
De notar que tanto a pré como a pós condição teem de ser elementos booleanos e devem-se referir à assinatura da função. Assim sendo estamos a definir que apenas o contrato $C$
é válido se a sua pré e pós condições devolverem $true$. A pré e pós condição podem ser vazias.\\

Neste sistema que difinimos a assinatura $C$ da função pode ter uma particularidade, que é a instânciação de um elemento que pertença a um determinado tipo.
Ou seja, pode-se dizer $$soma :: a \unif Int \rarrow b \unif Int \rarrow Int$$ para expressar que a função $soma$ recebe dois parametros $a$ e $b$ que são inteiros e devolve
um elemento do tipo inteiro.\\

Decidimos, por motivos de facilidade de leitura e afim de evitar a complexidade formal, não especificar o estado interno do sistema, como por exemplo o estado do Apache,
da base de dados entre outros componentes do sistema. Achamos que especificar isso não iria trazer nada de interessante ao que pretendemos mostrar aqui.
Assim, neste modelo formal apenas se vê de forma clara, os contratos que queremos que o nosso sitema tenha.\\

Começamos então por definir o contrato da função $login$ que permite a um determinado utilizador entrar no sistema. Este contrato estipula que recebendo um par
$Username \times Hash$ e um $SessionID$ devolve ou um $Error$ ou um novo $SessionID$ que associa o utilizador à sua sessão no sistema. Afim de haver provacidade
sobre os dados criticos do utilizador, como a password, decidimos apenas receber do lado do servidor a $Hash$ respectiva da sua palavra-passe, sendo esta hash
gerada do lado do cliente. Esta técnica não tem nada de novo, mas por incrível que pareça ainda há sistemas online que não usam este tipo de mecanismos.\\

\prop
{existsInDatabase(u)}
{login :: u \unif Username \times Hash \rarrow SessionID \rarrow Error + SessionID}
{ }

Aprenta-se de seguida o modelo de dados formal que o sistema usa. Consideramos que um $Exercicio$ tem um $Enunciado$ e um dicionário a relacionar $Input's$ com $Output's$,
um concurso tem um nome, um tipo e um conjunto de exercicios.\\

$\begin{array}{l}
data~Dict~a~b = (a \times b)^{*} \\
data~Exercicio = Exercicio~Enunciado~(Dict~Input~Output) \\
data~Contest =  Contest~Nome~Tipo~Exercicio^{*}
\end{array}$\\

Para criar um novo concurso, temos de assegurar que o utilizador que requisita este serviço é um professor, visto não nos interessar que alunos criem concursos.
Temos ainda de assegurar que o concurso que se vai criar tem no minimo um exercicio.\\

\prop
{ existeSession(s)  \wedge isProf(s) \wedge (notEmpty \circ getExercice)~c}
{createContest :: s \unif SessionID \rarrow c \unif Contest \rarrow 1}
{ (notEmpty \circ getDict)~c }

Para criar um novo exercicio, o utilizador tem de ser um professor e o exercicio em questão não pode ser repetido no sistema. Decidiu-se assim para evitar redundância
na informação que se tem armazenada.

\prop
{ existeSession(s)  \wedge isProf(s) \wedge (not \circ exist) (Exercicio e d)}
{createExercice :: s\unif SessionID \rarrow e \unif Enunciado \rarrow d \unif (Dict a b) \rarrow 1}
{ exerciceCreated(Exercicio e d) }

Pode-se ainda consultar os logs de um concurso especifico. Aqui apresentamos como argumento toda o concurso - $Contest$ para apenas evitar a definição evidente
de identificadores. É claro que a implementação desta acção, irá receber como parametro o identificador do concurso, como em outros parametros deste modelo formal.

\prop 
{ existeSession(s) \wedge isProf(s) \wedge contestIsClosed(c) }
{consultarLogsContest :: s \unif SessionID \rarrow c \unif Contest \rarrow LogsContest}
{}

De seguida mostra-se a especificação de efectuar um registo no concurso, queremos apenas que o concurso não esteja cheio.

\prop
{ existSession(s) \wedge contestNotFull(c)}
{registerOnContest :: s \unif SessionID \rarrow c\unif Contest \rarrow Credenciais}
{ }

Pode-se ainda sumbmeter um exercicio, o que implica que esta acção tenha como consequência devolver um relatório com os resultados interessantes para monstrar
ao participante de um concurso.

\prop
{ existeSession(s) \wedge exerciceExist(e) }
{ submitExercicio :: s \unif SessionID \rarrow e \unif Exercicio \rarrow res \unif Resolucao \rarrow rep \unif Report}
{ rep = geraReport s e res }

\subsubsection{Modelação de acções de selecção}
Temos acções do nosso sistema que envolvem a selecção de items nos forms ou então métodos que geram efeitos secundários, assim decidimos explicar aqui esse conjunto de
contractos. Estes métodos são interessantes de modelar porque, assim fica mais claro ver os parametros que recebemos para os concretizar.\\

Temos então a acção de escolha de um exercicio numa panoplia de exercicios disponiveis no concurso que o utilizador está actualmente inscrito e a participar.\\
Relembramos que o uso do tipo $1$ significa o tipo unitário, ou seja, queremos denotar que do ponto de vista formal esta operação não devolve nada, apenas altera o sistema.
Sistema esse que no inicio explicamos que por motivos de complexidade não tinha interesse expecificar formalmente.

\prop
{ existeSession(s) \wedge exerciceExist(e) }
{escolheExercicio :: s \unif SessionID \rarrow e \unif Exercicio \rarrow 1}
{ }

Temos ainda a escolha do concurso para um grupo que está já registado.

\prop
{ existSession(s) \wedge existContest(c) \wedge userRegistadoNoContest(s,c) }
{escolheConcursoJaRegistado :: s \unif SessionID \rarrow c \unif Contest \rarrow 1}
{ }

Mostramos de seguida o contracto da função que gera um relatório ao concorrente.

\prop
{ }
{geraReport :: e \unif Exercicio \rarrow res \unif Resolucao \rarrow Report}
{ }

A titulo de exemplo, da potencialidade deste tipo de modelação, servimo-nos agora da linguagem do Haskell para especificar a definição desta operação em maior detalhe.
Temos então as seguintes definições:

\begin{eqnarray*}
geraReportBugCompile :: Exercicio \rarrow Error \rarrow Report\\
geraReportBugCompare :: Exercicio \rarrow Errado \rarrow Report\\
geraReportNoBug :: Exercicio \rarrow Resolucao \rarrow Report\\
\\
execute :: Program \rarrow Exercicio \rarrow ResolucaoProposta\\
\end{eqnarray*}

Temos ainda que $ResolucaoProposta$ é a resolução submetida pelos concorrentes e aprenstenta o seguinte tipo:

$\begin{array}{l}
data~ResolucaoProposta = Dict~Input~Output
\end{array}$\\

Temos ainda a função que recebe uma resolução como input e devolve ora um programa pronto já compilado, ora um erro no caso de surgir algum problema na compilação.

\prop
{ }
{compile :: Resolucao \rarrow Error + Program}
{ }

E ainda uma função de comparação que recebe uma resolução e um exercicio e devolve se o Output pretendido é o mesmo que o que o programa submetido origina.

\prop
{ length(Exercicio)==length(ResolucaoProposta)}
{compare :: Exercicio \rarrow ResolucaoProposta \rarrow Certo + Errado}
{ }

\begin{lstlisting}[language=HaskellUlisses]
geraReport :: Exercicio -> Resolucao -> Report
geraReport exer res = do
	case compile res of
		(Left error) -> geraReportBugCompile error res
		(Right p) ->
			let resProps = execute p exer
			in case (compare exer resProps) of
				(Left certo) -> geraReportNoBug e res
				(Right errado) -> geraReportBugCompare errado res
\end{lstlisting}

Se quisessemos este código de um ponto de vista de composição de funções a sua conversão poderia ser fácilmente atingida pelas seguintes definições:

\begin{lstlisting}[language=HaskellUlisses]
geraReport exer res =
	compile res >>= \p -> compare exer (execute p exer)
		>>= \c -> geraReportNoBug exer res
\end{lstlisting}

%geraReport e res = compile res >>= \p \rarrow compare(e (execute p e)) >>= pageCerto)

Temos ainda o contrato da função que gera o relatório final baseando-se na participação de uma equipa num determinado concurso.

\prop
{ existSession(s) \wedge existContest(c) \wedge }
{geraFinalReport :: s \unif SessionID \rarrow c \unif Contest \rarrow Dict Exercicio Resolucao \rarrow Report}
{ }

\section{Modelo de Dados}\label{sec modedados}

Definiu-se que existirão três tipos de utilizadores: o administrador, o docente e o grupo.\\

\begin{itemize}
  \item Administrador - é a entidade com mais poder no sistema. É o único que pode criar contas do tipo docente. É caracterizado por:
    \begin{itemize}
      \item Nome de utilizador;
      \item Nome completo;
      \item Password
      \item e-Mail
    \end{itemize}
  \item Docente - entidade que tem permissões para criar concursos, exercícios, aceder aos resultados das submissões, (...). 
Os seus atributos coincidem com os do Administrador.
  \item Grupo - entidade que pode registar-se em concursos e submeter tentativas para os seus diferentes enunciados.
\end{itemize}

Decidiu-se que o sistema terá a noção de grupo, e um grupo não é mais do que um conjunto de concorrentes. No entanto,
o grupo é que possui as credenciais para entrar no sistema (nome de utilizador e password). 
Além disso também tem um nome pelo qual é identificado, um e-mail que será usado no caso de haver necessidade de se entrar em contacto
com o grupo, e um conjunto de .\\

Achamos importante incluir a informação de cada concorrente no grupo para, se possível, automatizar várias tarefas tais como lançamento de notas.
Cada concorrente é caracterizado pelo seu nome completo, número de aluno (se for o caso), e e-mail.\\

\begin{figure}[htbp]
\begin{center}
\includegraphics[width=0.9\textwidth]{Images/grupo-docente}
\caption{Modelo de dados - Grupo e Docente/Admin}\label{fig modedados-grupo-doc}
\end{center}
\end{figure}


Para finalizar vamos explicar em que consistem os concursos, enunciados e tentativas.

Um concurso, resumidamente, é um agregado de enunciados. 
Tem outras propriedades tais como um título, data de inicio e data de fim (período em que o concurso está disponível para que os grupos se registem), 
chave de acesso (necessária para o registo dos grupos), duração do concurso (tempo que o grupo tem para resolver os exercícios do concurso, 
a partir do momento que dá inicio à prova), e por fim, regras de classificação.\\
\\
Um enunciado é um exercício que o concorrente tenta solucionar. Como seria de esperar, cada exercício pode ter uma cotação diferente, 
logo o peso do enunciado é guardado no mesmo. 
Para cada enunciado existe também um conjunto de inputs e outputs, que servirão para verificar se o programa submetido está correcto. 
Contém ainda uma descrição do problema que o concorrente deve resolver, assim como uma função de avaliação, função esta que define como
 se verifica se o output obtido está de acordo com o esperado.\\
\\
Uma tentativa é a informação que é gerada pelo sistema, para cada vez que o grupo submete um ficheiro.\\
Além de conter dados sobre o concurso, enunciado e grupo a que pertence, a tentativa também contém o caminho para o código fonte do programa
submetido, data e hora da tentativa, dados referentes à compilação e um dicionário com os inputs esperados e os outputs gerados pelo programa
submetido.
\begin{figure}[htbp]
\begin{center}
\includegraphics[width=0.9\textwidth]{Images/concurso-enunciado}
\caption{Modelo de dados - Concurso, tentativa e enunciado}\label{fig modedados-conc-enunc}
\end{center}
\end{figure}

\section{Importação de dados}\label{sec xml}

Uma das funcionalidades requeridas ao nosso sistema é a importação de enunciados e tentativas no formato xml.
Esta funcionalidade será bastante útil para demonstrar e testar o sistema, sem que se tenha de criar manualmente os enunciados
 usando a interface gráfica, ou se tenha que submeter ficheiros com código fonte, de modo a serem geradas tentativas.
Os campos presentes no xml de cada uma das entidades, \textit{enunciado} e \textit{tentativa}, são praticamente os mesmos 
que estão descritos no modelo de dados das respectivas entidades.\\
\\
No xml do enunciado, não há nada muito relevante a acrescentar, além de que não contém um id para o enunciado, pois este será gerada 
automaticamente pelo sistema. Passamos agora a apresentar um exemplo do mesmo.
\\
\lstinputlisting{resources/enunciado.xml}


Quanto ao xml para a \textit{tentativa} há que realçar o facto de que o código fonte do programa vai dentro de uma tag xml.  Além da tag xml, o código fonte
terá de ir cercado de uma secção CDATA. Isto acontece para que o que o código fonte não seja processado com o restante xml que o contém.\\
\\

\lstinputlisting{resources/tentativa.xml}


Para que todos os dados contidos nos ficheiros xml possam ser facilmente validados, foram criados dois \textit{XML schema}.
Neste schema definimos quais as tags que devem existir em cada xml, o tipo de dados e até a gama de valores que serão contido por cada tag e
 a multiplicidade das tags .\\

Nesta fase inicial do projecto ainda não foram sempre especificados  os tipos de dados que serão contidos por cada tag.\\
No entanto, para alguns dos casos em que tal aconteceu apresentaremos alguns exemplos e explicações.\\
\\
No xsd referente ao \textit{enunciado} encontramos o elemento \textit{Peso}, que é um exemplo de uma tag que contém restrições.
O \textit{Peso} terá de ser um inteiro e terá um valor entre 0 e 100.

\begin{lstlisting}
<ed:element name="Peso" default="25">
  <ed:simpleType>
    <ed:restriction base="ed:integer">
      <ed:minInclusive value="0"/>
      <ed:maxInclusive value="100"/>
    </ed:restriction>
  </ed:simpleType>
</ed:element>
\end{lstlisting}
Já o elemento \textit{Linguagem} é também restringido, mas de uma forma ligeiramente diferente. A \textit{Linguagem} será uma string, mas
apenas poderá tomar um dos valores enumerados no xsd.\\

\begin{lstlisting}
<ed:element name="Linguagem" maxOccurs="unbounded">
    <ed:simpleType>
	<ed:restriction base="ed:string">
	    <ed:enumeration value="C"/>
	    <ed:enumeration value="Java"/>
	    <ed:enumeration value="Haskell"/>
	</ed:restriction>
    </ed:simpleType>
</ed:element>
\end{lstlisting}


No xsd para a \textit{tentativa} podemos evidenciar a multiplicidade das tags, ou seja, quantas vezes algumas delas se podem repetir.
Na \textit{tentativa}, existe um \textit{Dict}, que contém uma ou mais tags \textit{Teste}.
Para definirmos que possam existir mais de que uma tag \textit{Teste} dentro de \textit{Dict}, adicionamos o atributo \textit{maxOccurs},
na entidade \textit{Teste}, com o valor \textit{``unbounded''}. O valor mínimo não é necessário definir, porque é um por default.
\begin{lstlisting}
<tt:element name="Dict">
    <tt:complexType>
	<tt:sequence>
	    <tt:element name="Teste" maxOccurs="unbounded">
		<tt:complexType>
		    <tt:sequence>
			<tt:element name="Nome" type="tt:string"/>
			<tt:element name="Input" type="tt:string"/>
			<tt:element name="Output" type="tt:string"/>
		    </tt:sequence>
		</tt:complexType>
	    </tt:element>
	</tt:sequence>
    </tt:complexType>
</tt:element>
\end{lstlisting}

Para dar uma ideia mais geral sobre ambos os \textit{xml schema} criados, em vez de apresentarmos aqui ambos os ficheiros integralmente,
vamos antes expor os diagramas que o programa \textit{oxygen} constrói e coloca ao nosso dispor, pois pensamos que torna o entendimento do
schema muito mais simples.\\
\newpage
Desta forma aqui ficam os diagramas para o enunciado e para a tentativa:\\
 
\begin{figure}[htbp]
\begin{center}
\includegraphics[width=0.9\textwidth]{Images/enunciado_schema}
\caption{diagrama do schema para o enunciado}\label{fig xsd enunciado}
\end{center}
\end{figure}

\begin{figure}[htbp]
\begin{center}
\includegraphics[width=0.9\textwidth]{Images/tentativa_schema}
\caption{diagrama do schema para a tentativa}\label{fig xsd tentativa}
\end{center}
\end{figure}

\newpage
\part{Milestone II e III}
\chapter{Web Application} \label{chap webApp}
\minitoc
Neste capítulo vamos expôr alguns pormenores relacionados com a implementação do problema proposto.
Será explicada a maneira que encontramos para que seja seja cumprida a arquitectura que escolhemos e modelamos inicialmente.

\section{Criação de grupos, docentes e concorrenctes}\label{sec gdc}
O sistema permite que qualquer utilizador não registado se registe como grupo ~\ref{img:signup}, e associe a si um ou mais concorrentes. Para efectuar o registo terá de preencher o nome do grupo/docente, um e-mail válido e uma password que tenha entre 6 a 40 caractéres.\\
Este login é utilizado por todo o grupo, para participar nos mais variados concursos.\\

\begin{figure}[H]
\begin{center}
\includegraphics[width=0.45\textwidth]{Images/signup}
\caption{Página de registo}\label{img:signup}
\end{center}
\end{figure} 

Um concorrente é caracterizado por um nome, um número de aluno e um e-mail ~\ref{img:grupo}.

\begin{figure}[H]
\begin{center}
\includegraphics[scale = 0.6]{Images/grupo}
\caption{Dados de um grupo}\label{img:grupo}
\end{center}
\end{figure} 



As contas de docente só podem ser criados pelo administrador. Para simplificar o trabalho do administrador, o docente pode criar 
uma conta de grupo, à qual mais tarde será concedida privilégios de docente ~\ref{img:userToAdmin}.

\begin{figure}[H]
\begin{center}
\includegraphics[scale=0.60]{Images/userToAdmin}
\caption{Comandos necessários para tornar um utilizador sem privilégios num docente.}\label{img:userToAdmin}
\end{center}
\end{figure} 

\section{Linguagens de programação}\label{sec lps}
O nosso sistema é multilingue, ou seja, é possível submeter código fonte em várias linguagens de programação diferentes, desde que 
a linguagem tenha sido correctamente configurada por um docente.
Cada linguagem é caracterizada por uma séria de campos ~\ref{img:newLang}, os quais serão explicados de seguida:
\begin{itemize}
\item string de compilação: string que será executada quando se pretender compilar determinado código fonte. Esta string tem a
particularidade de no lugar em que é suposto conter o nome do ficheiro a compilar, contém \textit{\#\{file\}}.\\
Desta forma a string de compilação torna-se genérica, e idependente do nome do ficheiro a compilar.
exemplo: gcc -O2 -Wall \#\{file\}

\item string simples de execução: string utilizada para executar quando o código fonte foi compilado pela string de compilação.\\
exemplo 1:\\ 
- string de compilação: gcc -O2 -Wall \#\{file\}\\
- string de execução respectiva: ./a.out\\
exemplo 2:\\
- string de compilação: gcc -O2 -Wall \#\{file\} -o exec\\
- string de execução respectiva: ./exec\\

\item string complexa de execução: a necessidade de uma segunda string de execução surgiu quando tentamos preparar o sistema para receber makefiles (inicialmente apenas para C). Nestes casos, o nome do executável gerado pela compilação não é conhecido à partida.
Desta forma é necessário analisar o makefile, e só depois executar, tendo em conta a informação que retiramos do makefile.\\
Assim, e para a linguagem C, a string complexa de execução seria:\\
- ./\#\{file\}\\
em que \#\{file\} representa o nome do executável.

\begin{figure}[H]
\begin{center}
\includegraphics[scale=0.60]{Images/newLang}
\caption{Configuração de uma nova linguagem de programação no sistema}\label{img:newLang}
\end{center}
\end{figure} 

\end{itemize}

\section{Submissão de programas}\label{sec subm}
Sempre que acharem adequado, os grupos podem submeter os seus programas para serem avaliados. Para tal têm de escolher a linguagem de programação na qual resolveram o problema, de entre as disponíveis. E de seguida basta escolherem o ficheiro que pretendem submeter e carregar no botão de submissão.\\
No caso do administrador, pode ainda submeter tentativas no formato xml.

\begin{figure}[H]
\begin{center}
\includegraphics[scale=0.60]{Images/submissao}
\caption{Página de submissão de programas (vista de administrador)}\label{img:subm}
\end{center}
\end{figure} 

\section{Compilação}\label{sec comp}

Estando as linguagens de programação correctamente configuradas, a compilação torna-se bastante simples. Quando uma tentativa
é submetida no sistema, começamos por verificar se foi submetida apenas um ficheiro de código, ou um ficheiro comprimido.\\
Caso seja apenas um ficheiro, o sistema tenta compilar o código submetido, com a string de compilação da linguagem de programação em causa.\\
No caso de se tratar de um ficheiro comprimido, depois de o descomprimir, o sistema verifica se existe um makefile entre os ficheiros extraídos. Caso se verifique, é corrido o comando \textit{make}, e tenta retirar o nome do executável gerado, de mode a poder
ser usado na execução.

\section{Execução}\label{sec exec}
No fim da compilação, o sistema vai executar o programa uma vez para cada input. O processo de execução no caso de a compilação ter
sido feita à custa do makefile, é feita usando a string complexa de execução (sendo o nome do executável aquele que foi retirado do
makefile) . Se tal não tiver acontecido, é usada a string simples.\\
A execução pode ser abortada se ultrapassar o tempo máximo de execução, que é definido aquando da criação do enunciado em
questão. A título de demonstração, de seguida apresentamos a porção de código que "mata" o processo relativo à execução do programa,
caso ele demore mais de que o tempo máximo.

\begin{haskell}
    #thread que executa o a.out 
    out = "default";i=0
    exec = Thread.new do
      out = `#{execString} #{input}`
    end
    
    #thread que conta x segundos e dps termina a execucao do programa
    timer = Thread.new do
      sleep 5
      if exec.alive?
        Thread.kill(exec)
        i=1
        if params[:tentativa][:execStop] == false
          @erros += "Time out! Pelo menos a execucao de um dos inputs foi terminada por demorar demasiado tempo!"
        end
        params[:tentativa][:execStop] = true
      end
    end
    
    exec.join
    if timer.alive?
      Thread.kill(timer)
    end
    timer.join

\end{haskell}

\section{Guardar resultados}\label{sec res}
Para cada input do enunciado em questão, o programa é executado uma vez. O seu output é comparado com o output esperado e 
é guardada uma entrada na base de dados com a percentagem de testes nos quais o programa teve sucesso.\\
No caso de o código não compilar, ou da execução do programa demorar mais tempo do que o máximo previsto pelo docente quando 
criou o enunciado, estas informações são também guardadas na base de dados.\\
Além de se guardarem todas as tentativas, a melhor é também guardada numa tabela à parte, para que o melhor resultado para cada
enunciado seja de fácil acesso.\\
A qualquer momento o grupo pode consultar os dados relativos às suas últimas tentativas ou às últimas tentativas de todos os participantes ~\ref{img:tentativas}, e também os seus melhores resultados.\\

\begin{figure}[H]
\begin{center}
\includegraphics[scale=0.60]{Images/tentativas}
\caption{Página aonde pode consultar as tentativas (vista das tentativas de todos os utilizadores)}\label{img:tentativas}
\end{center}
\end{figure} 
\newpage
\chapter{Métricas}
\minitoc

Existem diversas métricas, com objectivos diferentes, para analisar um projecto de \emph{software}. Essas métricas podem ser vistas como diferentes 'lentes' com as quais olhamos para um software. 
Neste capítulo, pretende-se mostrar a investigação que foi realizada relativamente a este tema, começando primeiro por definir alguns conceitos e depois dividir as métricas 
por categorias. Cada categoria será estruturada da mesma maneira, indicada mais à frente.\\

Assim, encarou-se a análise a um software como sendo uma área que se divide em dois ramos, a \textbf{Análise Estática} e a \textbf{Análise Dinâmica}.\\

\section{Análise Estática}
A Análise Estática é olhar para um programa sob o ponto de vista do seu código ou ficheiro já compilado, e retirar conclusões sobre as suas características, 
sem nunca recorrer à sua execução ou análise de resultados da execução.
Ainda relativamente à Análise Estática, temos essencialmente duas maneiras de olhar para o software. Podemos ver este tendo em conta a qualidade do ficheiro objecto 
produzido ( o código máquina que ira correr, p. ex: em java seria o \emph{bytecode}) ou então tendo única e exclusivamente como objecto de observação os ficheiros de 
texto correspondentes ao código que compõe o programa.
De notar que a Análise Estática é sempre relativa ao código do programa, ou seja, até mesmo uma análise que tenha em vista a qualidade do ficheiro objecto vai 
ser feita sobre o código do programa.\\

Assim sendo, no que diz respeito à qualidade do ficheiro objecto produzido, temos:

\paragraph{Syntax checking} é um programa ou parte de um programa que tenta atestar a correcção da linguagem escrita.

\paragraph{Type checking} é o processo de verificacao dos tipos de dados num software que visa garantir a restrição 
no que diz respeito aos tipos, implicando assim maior qualidade do software produzido e menos probabilidade de acontecerem erros aquando da execução. 
Cada vez mais as linguagens recentes apresentam este tipo de sistemas, o que levam a que muitos dos erros ocorram em tempo de compilação, ou seja: 
completamente ainda a tempo de serem corrigidos pelos programadores, por exemplo: \texttt{Haskell, C++0x, JAVA6}. 
Estas linguagens apresentam um sistema de tipos forte o que garante este processo. 

\paragraph{Decompilation} é o processo de pegar num ficheiro objecto e tentar inferir ou descobrir o seu código fonte que o originou. 
Designa-se assim porque é o inverso do processo de compilação. 
Com a ajuda deste tipo de análise consegue-se obter, entre outras coisas, os algoritmos alto nível do código máquina em questão.\\

No que diz respeito à análise da qualidade do código como produto final temos as seguintes metodologias:

\paragraph{Code metrics} é uma vasta área que se dedica a análise do código em si para tirar conclusão acerca da sua qualidade, estabilidade e manutenção.

\paragraph{Style checking} funciona como uma análise para verificar determinadas regras que à partida se acreditam como boas na produção de código. 
Estas regras podem ser relativas a identação,  existência de ficheiros \texttt{README} e de documentação.

\paragraph{Verification reverse engineering}, o método que serve para verificar se a implementação de um determinado sistema cumpre a sua especificação.

O objectivo deste trabalho é puramente analisar estaticamente um programa, relativamente às metricas de código e eventualmente relativamente ao estilo também.
Mesmo assim este tema tão vasto deixou-nos com motivação para conhecer o que é este mundo da análise de software.\\

\section{Análise Dinâmica}
Outros tipos de análises existentes são as chamadas análises dinâmicas, estas pegam numa peça de \emph{software} e não tendo em conta, 
nem se preocupando com o código que a constitui, executam simplesmente o programa e analisam exaustivamente sob vários prismas o seu comportamento.
De seguida vamos dissertar sobre alguns destes métodos e práticas que existem para analisar a execução de um programa.\\

\paragraph{Log analysis} é o método que consiste em pesquisar (automaticamente ou manualmente) 
os ficheiros de log produzidos por um determinado \emph{software}, para perceber o que este está a fazer. 
Este tipo de análise muitas das vezes é feita a programas muito complexos e extensos que comunicam com o mundo real (rede, stdin, mundo IO).
Um exemplo de quem faz este tipo de análise são os administradores de sistemas.

\paragraph{Testing} é investigar o comportamento de um software através de uma bateria de testes que podem ter em consideração um determinado uso num caso que pode ser real. 
Geralmente, este tipo de análise simula os casos extremos a que o \emph{software} pode ir, porque se acredita empiricamente que ao ter sucesso em situações extremas, 
há-de ter sucesso nos restantes casos.
O que se pretende obter com este tipo de análise é o aumento na confiança de que o programa está a fazer o que é suposto, por parte de quem fabrica o produto.

\paragraph{Debugging} é um método que ajuda as pessoas a terem conhecimento do que determinado \emph{software} está a fazer. Este método geralmente é usado ainda numa 
fase inicial do produto, quando está a ser desenvolvido pelos programadores. É um bom método de detectar defeitos, falhas ou pequenos \emph{bugs} no \emph{software}.

\paragraph{Instrumentation} é o método de monitorizar e medir o nível de performance de um determinado produto.

\paragraph{Profiling} é a investigação sobre o comportamento de um programa aquando a sua execução, usando para isso informações do género recursos computacionais. 
Este tipo de análise é útil para por exemplo efectuar gestão de memória.

\paragraph{Benchmarking} é o processo de comparar o processo do utilizador com os processos conhecidos de outros, de modo a obter conhecimento 
sobre as melhores práticas efectuadas na indústria.

\section{Métricas de qualidade de \emph{software}}

A presença de testes num determinado programa de \emph{software}, leva a que esse artefacto ganhe pontos no que diz respeito à análise estática sob o ponto de vista da 
qualidade, porque, como dissemos anteiormente, a presença de testes num projecto de \emph{software} leva a que tenhamos mais confiança neste.
Existem algumas fórmulas que nos dão alguns indicadores numerários sobre o nível desta confiança. De seguida são apresentadas algumas sobre a cobertura de testes.\\

\[ \emph{\text{Line Coverage}} = \frac{\emph{\text{Nr of test lines}}}{\emph{\text{nr of tested lines}}} \] \\

\[ \emph{\text{Decision coverage}} = \frac{\emph{\text{Nr of test methods}}}{\emph{\text{Sum of McCabe complexity}}} \] \\

\[ \emph{\text{Test granularity}} = \frac{\emph{\text{Nr of test lines}}}{\emph{\text{nr of tests}}} \] \\

\[\emph{\text{Test efficiency}} = \frac{\emph{\text{Decision coverage}}}{\emph{\text{line coverage}}} \] \\


Podemos sempre aumentar a nossa precisão na análise se considerarmos apenas as linhas que contêem código e não as linhas em branco ou linhas com apenas um caracter, 
como por exemplo as aberturas de blocos em \texttt{C, JAVA}.
Ainda podemos também considerar não apenas o numero total de linhas mas também o número de métodos/funções testados.

De notar que as formulas atrás descritas podem ser modificadas para obter outros tipos de métricas, não só referentes a testes, como por exemplo:\\

\[\emph{\text{Code granularity}} = \frac{\emph{\text{Nr of lines}}}{\emph{\text{Nr of (methods/functions)}}} \] \\

Como já referimos anteriormente, a análise que se pretende, por agora, é essencialmente estática. Assim sendo, segue-se uma lista de técnicas no que diz respeito à análise
 estática, estando estruturada da seguinte maneira: duas secções, uma para \emph{\textbf{Patterns}} e outra para \emph{\textbf{Métricas de qualidade}}

\subsection{Patterns}

\newpage
\chapter{Interface pelo terminal}
\minitoc

Tendo em vista a facilidade, para alguns utilizadores, em manusear um sistema por um terminal, decidiu-se criar uma interface para a aplicação desenvolvida. Esta interface, 
ainda em fase de desenvolvimento, vai permitir, essencialmente, trabalhar com a base de dados do sistema. Os objectivos passam por consultar listas de determinadas entidades, 
desde \texttt{enunciados} a utilizadores do sistema. De realçar que este modo de comunicação com o sistema apenas é utilizado pelos \textbf{administradores}\\

\section{Perl}

Como linguagem de desenvolvimento para esta interface, decidiu-se usar \texttt{Perl} devido à rapidez de implementação (visto que a criação de uma interface pelo terminal não 
constituía um dos principais objectivos) e a vasta diversificação de módulos existentes para auxílio ao desenvolvimento. Desses módulos, destaca-se o uso do módulo 
\texttt{DBIx::Class}, um módulo de comunicação a base de dados (apresentado durante as aulas de EL::PLN), 
que basicamente representa em classes cada tabela existente na base de dados, transformando também simples \texttt{querys} em métodos sobre as tabelas.\\

Tem-se em vista também a utilização de um módulo que use a biblioteca do sistema \texttt{Readline} e {\large{Falta: readline, hash md5 para ser compativel com ruby ...}}

\section{Menus e exemplos}

Na fase actual desta interface, o utilizador desta interface terá pela frente um menu principal (Figura~\ref{img:menuprinc}). \\

\begin{figure}[H]
\begin{center}
\includegraphics[width=0.45\textwidth]{Images/menuPrinc}
\caption{Menu Principal}\label{img:menuprinc}
\end{center}
\end{figure} 

Cada escolha desse menu representa as principais acções que se podem efectuar com uma base de dados: 

\begin{itemize}
 \item A listagem de elementos
 \item A procura de certos elementos e posterior actualização dos mesmos
 \item A inserção de novos elementos
\end{itemize}

Assim, caso o utilizador escolha a primeira opção, será levado para um novo menu contendo as 5 principais entidades deste sistema: Os \texttt{Users}; os \texttt{Enunciados}; as \texttt{Concorrentes}; os \texttt{Concursos}; e as \texttt{Linguagens} disponíveis para responder em cada concurso. A título de exemplo, caso o utilizador quisesse saber 
as linguagens disponíveis, a informação seria apresentada da seguinte forma(Figura~\ref{img:linguagens}):\\

\begin{figure}[H]
\begin{center}
\includegraphics[width=0.45\textwidth]{Images/linguagens}
\caption{Linguagens disponíveis}\label{img:linguagens}
\end{center}
\end{figure} 

O utilizador também poderia procurar por um único elemento. Como se pode ver na Figura~\ref{img:zelladouglas}, dando um nome de utilizador, 
seria retornada a informação sobre esse sujeito. Caso pretendesse, o utilizador poderia posteriormente proceder à alteração do mesmo.\\

\begin{figure}[H]
\begin{center}
\includegraphics[width=0.65\textwidth]{Images/zacarias}
\caption{Procura e alteração do \texttt{user} Zella Douglas}\label{img:zelladouglas}
\end{center}
\end{figure} 

\newpage
\chapter{Scripts de avaliação}
\minitoc

A ideia de usar scripts auxiliares prende-se com o facto de muitas pequenas operações conseguem ser muito simplificadas com a rápida implementação
de um script que ajue a resolver o problema em questão. Como linguagem principal usamos essencialmente o \textrm{Perl} por ser uma linguagem fundamental para esta \textrm{UCE},
mas porque também estarmos satisfeitos com as potencialidades únicas que apresenta.\\
Algumas vezes deparámos-nos com pequenos problemas onde não se justifica o uso de \textrm{Ruby} ou \textrm{Haskell} por trazer mais complexidade e baixar o ritmo ao
desenvolvimento desta aplicação.

\section{Geração de imagens}
Foi desenvolvido um script que usa essencialmente o módulo \textrm{GD} do \textrm{Perl} para gerar estatisticas relativas ao número de linhas do projecto submetido.
Este script (count.pl) encontra-se muito bem documentado, acompanhado de um \textrm{README} onde é explicado o seu funcionamento:

\begin{code_files}
perl count.pl -open <dirPath> [-verbose] [-separated | -allTogether]
    [-percent] [-bars | -pie] -out <fileNamePrefix>
\end{code_files}

Este script na sua utilização minima pode ser executado da seguinte maneira:
\begin{code_files}
perl count.pl -open ~/projecto -out projecto
\end{code_files}

Isto irá pesquizar recursivamente na pasta \textrm{\textasciitilde{}/projecto} todos os ficheiros de várias linguagens de programação e produzir $3$ imagens, todas elas com o
prefixo \textrm{projecto}.

\begin{code_files}
projecto_LinesPerLanguage.png
projecto_FilesPerLanguage.png
projecto_RatioFilesLines.png
\end{code_files}

Cada uma destas $3$ imagens contem informação relativa à totalidade do projecto, a primeira imagem (\ref{fig:linesperlanguage}) mostra a quantidade de linhas
de código e de comentários relativamente a todas as linguagens presentes no projecto. Desta forma conseguimos ter uma noção do impacto que cada linguagem
tem para o projecto final e a quantidade de documentação que existe relativamente a cada linguagem.

\begin{figure}[htbp]
\begin{center}
\includegraphics[width=0.9\textwidth]{Images/projecto_LinesPerLanguage.png}
\caption{Linhas por Linguagem}\label{fig:linesperlanguage}
\end{center}
\end{figure}

Outra imagem que a execução do comando anterior gera é a quantidade de ficheiros por linguagem, assim sendo temos uma noção da modularidade que existe
em cada utilização de cada uma das linguagens utilizadas.

\begin{figure}[htbp]
\begin{center}
\includegraphics[width=0.9\textwidth]{Images/projecto_FilesPerLanguage.png}
\caption{Ficheiros por Linguagem}\label{fig:filesperlanguage}
\end{center}
\end{figure}

A última imagem gerada é um rácio entre o número de ficheiros e o número de linhas para cada linguagem. Assim conseguimos ter uma
noção do número médio de
linhas que constituí cada um dos ficheiros do projecto.

\begin{figure}[htbp]
\begin{center}
\includegraphics[width=0.9\textwidth]{Images/projecto_RatioFilesLines.png}
\caption{Linhas por Linguagem}\label{fig:ratiofileslines}
\end{center}
\end{figure}

Como se pode verificar, todas estas estatísticas são meramente um indicador e nada de maior se pode concluir, sem ser apenas ter uma
noção global da quantidade de linhas
e o uso de quais linguagens relativamente a um projecto.

Caso queiramos podemos ainda gerar as mesmas imagens, mas em percentagem. Como é obvio não faz sentido gerar percentagens
sobre rácios, assim quando o utilizador
emite o comando:

\begin{code_files}
perl count.pl -open ~/projecto -out projecto -percent
\end{code_files}

geramos apenas a percentagem para as duas primeiras imagens.\\

Uma outra utilização interessante deste script permite a geraçã de uma única imagem com o intuíto de ter um resumo de todo o projecto
submetido numa única imagem, como
se a imagem que sumariza a utilização de linhas do projecto.\\
Para gerar este ficheiro precisamos de fazer:
\begin{code_files}
perl count.pl -open ~/projecto -out projecto -all
\end{code_files}
Por defeito quando utilizamos esta flag, todos os resultados que vão aparecer irão ser em percentagem, visto que podemos ter uma
grande disparidade de números de
linhas entre várias linguagens. Obtemos assim a imagem \ref{fig:projectlanguages}:

\begin{figure}[htbp]
\begin{center}
\includegraphics[width=0.9\textwidth]{Images/projecto_projectLanguages.png}
\caption{Visão global do projecto}\label{fig:projectlanguages}
\end{center}
\end{figure}

Este script é ainda capaz de gerar pie charts em vez de gráficos de barras.

\section{Makefile}
O nosso sistema de submissão permite ao utilizador submeter um \textrm{makefile} afim de facilitar a compilação do código do seu
projecto, caso a compilação deste seja mais
exigente do que uma trivial passagem pelo \textrm{gcc}. Assim é necessário extrair do \textrm{makefile} a informação sobre o ficheiro
objecto que este vai gerar aquando da
execução do comando \textrm{make}.\\
Servindo-nos do módulo \textrm{Perl} \textrm{Makefile::Parser} conseguimos, com poucas linhas extraír o nome do binário que irá ser
gerado pelo \textrm{makefile}.\\

Muitas das vezes, são scripts com a simplicidade que este apresenta que fazem a diferença e que mostram o verdadeiro poder de estar à
vontade numa linguagem de utilização rápida como sao as de scripting e nomeadamente o \textrm{Perl}.\\

Pretendemos melhorar este script e não o damos como terminado.
\section{Cloning}
Um dos objectivos finais para este projecto é que o sistema tenha um mecanismo de detecção de clones.\\
Apesar de ainda não implementado no sistema, o nosso trabalho nesse sentido já começou. Criamos um script na linguagem \textit{Perl},
a linguagem escolhida prende-se com o facto de grande parte das tarefas realizadas pelo script serem de processamento de texto.\\
De seguida vamos descrever passo a passo o que o script faz actualmente, tendo em conta que ainda não está preparado para ser
utilizado no sistema:\\
\begin{itemize}
\item recebe um ficheiro por parâmetro
\item executa o comando \textit{ctags -x *.c}, cujo output contém entre outras coisas, o nome das funções que existem nos ficheiros c
encontrados, e as linhas referentes a cada um
\item a partir do output do ctags retira as linhas que identificam o ínicio de cada função
\item lê o ficheiro recebido como parâmetro, divide-o por funções, guardando as linhas de cada uma, numa posição da hash
\item analisa cada posição da hash e altera-a, retirando todos os comentários e espaços, substituíndo todas as variáveis por \textit{var},
todas as strings por \textit{S} e todos os números por 1
\item de seguida lê um segundo ficheiro e trata-o da mesma forma que o primeiro
\item por fim compara cada posição da primeira hash com todas as posições da segunda. Se houver casos em que as duas são iguais,
imprime para o STDOUT um aviso
\end{itemize}

Temos noção de que o script precisa de algumas reformulações para a sua integração com o sistema. Para começar, em vez de receber
apenas um ficheiro por parâmetro deverá receber o path de dois ficheiros. Além de preparamos o script para ser integrado no sistema 
também é necessário afiná-lo de modo aque não existam muitos falsos positivos.\\
Estas entre outras, serão algumas das alterações ao script feitas para a próxima fase, que garantirá ao sistema a ferramenta necessária
para a detecção de clones.








\newpage
\chapter{Front-end}
\minitoc
Como já tinhamos explicado na primeira milestone, decidimos que o nosso sistema vai tentar suportar ao máximo avaliação de métricas sobre código C.
Seria muito interessante suportar outras, mas acreditamos nesta altura que se o nosso sistema for extendido para suportar a avaliação de outras linguagens que não o C
então deveriamos recorrer a ferramentas externas que fizessem algum trabalho por nós.\\
Relativamente ao Front-end que utilizamos, ele está feito em Haskell e foi um GSoc (Google Summer of Code) feito em 2008, chama-se Language.C.
Este pacote de software apresenta um completo e bem testado parser e pretty printer para a definição da linguagem \textrm{C99} e ainda muitas das \textrm{GNU extensions}.\\

A nossa ideia é pegar em toda a investigação e trabalho dedicado à análise e descoberta de métricas, que estão descritas no Capítulo \ref{chap:metricas}, e implementa-las
utilizando este Front-end.\\

Inicialmente decidimos partir para a exploração da linguagem (dos tipos de dados) que estavam definidos neste parser. Rápidamente encontramos a AST da linguagem \textrm{C99}
e assim descobrimos que a linguagem C não é assim tão grande como estariamos à espera, como podemos ver no Apêndice \ref{chap:ast}.

\section{Estudo do Front-End}
Todos os tipos deste parser estão munidos de um \textrm{NodeInfo}, que nada mais é do que a informação relativa ao ficheiro, número de linha e coluna onde apareceu
esta derivação.\\
Um ficheiro em linguagem \textrm{C99} é representado como uma lista de declarações externas que pode ser uma declaração ou uma definição de função como podemos ver
na secção \ref{chap:extdefin}.

\begin{haskell}
data CTranslUnit = CTranslUnit [CExtDecl] NodeInfo
data CExtDecl = CDeclExt CDecl
              | CFDefExt CFunDef
			  | CAsmExt CStrLit
\end{haskell}


\newpage
\section{Conclusion and Future Work}\label{conc}




\appendix
\newpage
\chapter{AST da Linguagem \textrm{C99}}\label{chap:ast}
\minitoc
Esta definição de AST é parte da original descrita no Apendix A do \cite{Kernighan:1988:CPL:576122}, excluímos algumas definições lexicas
como por exemplo a definição de digito ou de alfanumerico por ser fácil perceber o que se quer dizer com isso e para não alongar muito mais este apêndice.
\section{Gramática Lexica}
\subsection{Elementos Lexicos}

\begin{code_files}
token:
	keyword
	identifier
	constant
	string-literal
	punctuator
preprocessing-token:
	header-name
	identifier
	pp-number
	character-constant
	string-literal
	punctuator
	each non-white-space character that cannot be one of the above
\end{code_files}

\subsection{Keywords}
\begin{code_files}
keyword:
	one of, auto, break, case, char, const
	continue, default, do, double, else
	enum, extern, float, for, goto, if
	inline, int, long, register, restrict
	return, short, signed, sizeof, static
	struct, switch, typedef, union
	unsigned, void, volatile, while
	_Bool, _Complex, _Imaginary
\end{code_files}

\subsection{Identificadores}
\begin{code_files}
identifier:
	identifier-nondigit
	identifier identifier-nondigit
	identifier digit
identifier-nondigit:
	nondigit
	universal-character-name
	other implementation-defined characters
\end{code_files}

\subsection{Universal character names}
\begin{code_files}
universal-character-name:
	\u hex-quad
	\U hex-quad hex-quad
hex-quad:
	hexadecimal-digit hexadecimal-digit
	hexadecimal-digit hexadecimal-digit
\end{code_files}

\subsection{Constantes}
\begin{code_files}
constant:
	integer-constant
	floating-constant
	enumeration-constant
	character-constant
integer-constant:
	decimal-constant integer-suffixopt
	octal-constant integer-suffixopt
	hexadecimal-constant integer-suffixopt
decimal-constant:
	nonzero-digit
		decimal-constant digit
octal-constant:
	0
	octal-constant octal-digit
hexadecimal-constant:
	hexadecimal-prefix hexadecimal-digit
	hexadecimal-constant hexadecimal-digit
hexadecimal-prefix: one of
	0x 0X
integer-suffix:
	unsigned-suffix long-suffixopt
	unsigned-suffix long-long-suffix
	long-suffix unsigned-suffixopt
	long-long-suffix unsigned-suffixopt
unsigned-suffix: one of
	u U
long-suffix: one of
	l L
long-long-suffix: one of
	ll LL
floating-constant:
	decimal-floating-constant
	hexadecimal-floating-constant
decimal-floating-constant:
	fractional-constant exponent-partopt floating-suffixopt
	digit-sequence exponent-part floating-suffixopt

hexadecimal-floating-constant:
	hexadecimal-prefix hexadecimal-fractional-constant
	binary-exponent-part floating-suffixopt
	hexadecimal-prefix hexadecimal-digit-sequence
	binary-exponent-part floating-suffixopt
fractional-constant:
	digit-sequenceopt . digit-sequence
	digit-sequence .
exponent-part:
	e signopt digit-sequence
	E signopt digit-sequence
sign: one of
	+
digit-sequence:
	digit
	digit-sequence digit
hexadecimal-fractional-constant:
	hexadecimal-digit-sequenceopt .
	hexadecimal-digit-sequence
	hexadecimal-digit-sequence .
binary-exponent-part:
	p signopt digit-sequence
	P signopt digit-sequence
hexadecimal-digit-sequence:
	hexadecimal-digit
	hexadecimal-digit-sequence hexadecimal-digit
floating-suffix: one of
	f l F L
enumeration-constant:
	identifier
character-constant:
	' c-char-sequence '
	L' c-char-sequence '

c-char-sequence:
	c-char
	c-char-sequence c-char
c-char:
	any member of the source character set except
	the single-quote ', backslash \, or new-line character
	escape-sequence
escape-sequence:
	simple-escape-sequence
	octal-escape-sequence
	hexadecimal-escape-sequence
	universal-character-name
simple-escape-sequence: one of
	\' \" \? \\
	\a \b \f \n \r \t
	\v
octal-escape-sequence:
	\ octal-digit
	\ octal-digit octal-digit
	\ octal-digit octal-digit octal-digit
hexadecimal-escape-sequence:
	\x hexadecimal-digit
	hexadecimal-escape-sequence hexadecimal-digit
\end{code_files}

\subsection{String literals}
\begin{code_files}
string-literal:
	" s-char-sequenceopt "
	L" s-char-sequenceopt "
s-char-sequence:
	s-char
	s-char-sequence s-char
s-char:
	any member of the source character set except
	the double-quote ", backslash \, or new-line character
	escape-sequence
\end{code_files}

\subsection{Header names}
\begin{code_files}
header-name:
	< h-char-sequence >
	" q-char-sequence "
h-char-sequence:
	h-char
	h-char-sequence h-char
h-char:
	any member of the source character set except
	the new-line character and >
q-char-sequence:
	q-char
	q-char-sequence q-char
q-char:
	any member of the source character set except
	the new-line character and "
\end{code_files}

\section{Phrase structure grammar}
\subsection{Expressions}
\begin{code_files}
primary-expression:
	identifier
	constant
	string-literal
	( expression )
postfix-expression:
	primary-expression
	postfix-expression [ expression ]
	postfix-expression ( argument-expression-listopt )
	postfix-expression . identifier
	postfix-expression -> identifier
	postfix-expression ++
	postfix-expression -( type-name ) { initializer-list }
	( type-name ) { initializer-list , }
argument-expression-list:
	assignment-expression
	argument-expression-list , assignment-expression
unary-expression:
	postfix-expression
	++ unary-expression
	-- unary-expression
	unary-operator cast-expression
	sizeof unary-expression
	sizeof ( type-name )
unary-operator: one of
	& * + - ~
	!
cast-expression:
	unary-expression
	( type-name ) cast-expression
multiplicative-expression:
	cast-expression
	multiplicative-expression * cast-expression
	multiplicative-expression / cast-expression
	multiplicative-expression % cast-expression

additive-expression:
	multiplicative-expression
	additive-expression + multiplicative-expression
	additive-expression - multiplicative-expression
shift-expression:
	additive-expression
	shift-expression << additive-expression
	shift-expression >> additive-expression
relational-expression:
	shift-expression
	relational-expression
	relational-expression
	relational-expression
	relational-expression
	<
	>
	<=
	>=
	shift-expression

equality-expression:
	relational-expression
	equality-expression == relational-expression
	equality-expression != relational-expression
AND-expression:
	equality-expression
	AND-expression & equality-expression
exclusive-OR-expression:
	AND-expression
	exclusive-OR-expression ^ AND-expression
inclusive-OR-expression:
	exclusive-OR-expression
	inclusive-OR-expression | exclusive-OR-expression
logical-AND-expression:
	inclusive-OR-expression
	logical-AND-expression && inclusive-OR-expression
logical-OR-expression:
	logical-AND-expression
	logical-OR-expression || logical-AND-expression
conditional-expression:
	logical-OR-expression
	logical-OR-expression ? expression : conditional-expression
assignment-expression:
	conditional-expression
	unary-expression assignment-operator assignment-expression
assignment-operator: one of
	= *= /= %= += -=
	<<= >>= &= ^= |=

expression:
	assignment-expression
	expression , assignment-expression
constant-expression:
	conditional-expression
\end{code_files}

\subsection{Declarations}\label{chap:decl}
\begin{code_files}
declaration:
	declaration-specifiers init-declarator-listopt ;
declaration-specifiers:
	storage-class-specifier declaration-specifiersopt
	type-specifier declaration-specifiersopt
	type-qualifier declaration-specifiersopt
	function-specifier declaration-specifiersopt
init-declarator-list:
	init-declarator
	init-declarator-list , init-declarator
init-declarator:
	declarator
	declarator = initializer
storage-class-specifier:
	typedef
	extern
	static
	auto
	register

type-specifier:
	void, char, short, int, long
	float, double, signed, unsigned
	_Bool, _Complex, struct-or-union-specifier
	enum-specifier, typedef-name, *

struct-or-union-specifier:
	struct-or-union identifieropt { struct-declaration-list }
	struct-or-union identifier
struct-or-union:
	struct
	union
struct-declaration-list:
	struct-declaration
	struct-declaration-list struct-declaration
struct-declaration:
	specifier-qualifier-list struct-declarator-list ;
specifier-qualifier-list:
	type-specifier specifier-qualifier-listopt
	type-qualifier specifier-qualifier-listopt
struct-declarator-list:
	struct-declarator
	struct-declarator-list , struct-declarator
struct-declarator:
	declarator
	declaratoropt : constant-expression

enum-specifier:
	enum identifieropt { enumerator-list }
	enum identifieropt { enumerator-list , }
	enum identifier
enumerator-list:
	enumerator
	enumerator-list , enumerator
enumerator:
	enumeration-constant
	enumeration-constant = constant-expression
type-qualifier:
	const
	restrict
	volatile
function-specifier:
	inline
declarator:
	pointeropt direct-declarator
direct-declarator:
	identifier
	( declarator )
	direct-declarator [ type-qualifier-listopt assignment-expressionopt ]
	direct-declarator [ static type-qualifier-listopt assignment-expression ]
	direct-declarator [ type-qualifier-list static assignment-expression ]
	direct-declarator [ type-qualifier-listopt * ]
	direct-declarator ( parameter-type-list )
	direct-declarator ( identifier-listopt )
pointer:
	* type-qualifier-listopt
	* type-qualifier-listopt pointer
type-qualifier-list:
	type-qualifier
	type-qualifier-list type-qualifier
parameter-type-list:
	parameter-list
	parameter-list , ...

parameter-list:
	parameter-declaration
	parameter-list , parameter-declaration
parameter-declaration:
	declaration-specifiers declarator
	declaration-specifiers abstract-declaratoropt
identifier-list:
	identifier
	identifier-list , identifier
type-name:
	specifier-qualifier-list abstract-declaratoropt
abstract-declarator:
	pointer
	pointeropt direct-abstract-declarator
direct-abstract-declarator:
	( abstract-declarator )
	direct-abstract-declaratoropt [ type-qualifier-listopt
	assignment-expressionopt ]
	direct-abstract-declaratoropt [ static type-qualifier-listopt
	assignment-expression ]
	direct-abstract-declaratoropt [ type-qualifier-list static
	assignment-expression ]
	direct-abstract-declaratoropt [ * ]
	direct-abstract-declaratoropt ( parameter-type-listopt )

typedef-name:
	identifier
initializer:
	assignment-expression
	{ initializer-list }
	{ initializer-list , }
initializer-list:
	designationopt initializer
	initializer-list , designationopt initializer
designation:
	designator-list =

designator-list:
	designator
	designator-list designator
designator:
	[ constant-expression ]
	. identifier
\end{code_files}

\subsection{Statements}
\begin{code_files}
statement:
	labeled-statement
	compound-statement
	expression-statement
	selection-statement
	iteration-statement
	jump-statement
labeled-statement:
	identifier : statement
	case constant-expression : statement
	default : statement
compound-statement:
	{ block-item-listopt }
block-item-list:
	block-item
	block-item-list block-item
block-item:
	declaration
	statement
expression-statement:
	expressionopt ;
selection-statement:
	if ( expression ) statement
	if ( expression ) statement else statement
	switch ( expression ) statement

iteration-statement:
	while ( expression ) statement
	do statement while ( expression ) ;
	for ( expressionopt ; expressionopt ; expressionopt ) statement
	for ( declaration expressionopt ; expressionopt ) statement
jump-statement:
	goto identifier ;
	continue ;
	break ;
	return expressionopt ;
\end{code_files}

\subsection{External definitions}\label{chap:extdefin}
\begin{code_files}
translation-unit:
	external-declaration
	translation-unit external-declaration
external-declaration:
	function-definition
	declaration
function-definition:
	declaration-specifiers declarator declaration-listopt compound-statement
declaration-list:
	declaration
	declaration-list declaration
\end{code_files}

\section{Preprocessing directives}
\begin{code_files}
preprocessing-file:
	groupopt
group:
	group-part
	group group-part
group-part:
	if-section
	control-line
	text-line
	# non-directive
if-section:
	if-group elif-groupsopt else-groupopt endif-line
if-group:
	# if
	constant-expression new-line groupopt
	# ifdef identifier new-line groupopt
	# ifndef identifier new-line groupopt
elif-groups:
	elif-group
	elif-groups elif-group
elif-group:
	# elif constant-expression new-line groupopt

else-group:
	# else new-line groupopt

endif-line:
	# endif new-line

control-line:
	# include pp-tokens new-line
	# define identifier replacement-list new-line
	# define identifier lparen identifier-listopt )
	replacement-list new-line
	# define identifier lparen ... ) replacement-list new-line
	# define identifier lparen identifier-list , ... )
	replacement-list new-line
	# undef identifier new-line
	# line pp-tokens new-line
	# error pp-tokensopt new-line
	# pragma pp-tokensopt new-line
	# new-line
text-line:
	pp-tokensopt new-line
non-directive:
	pp-tokens new-line
lparen:
	a ( character not immediately preceded by white-space
replacement-list:
	pp-tokensopt

pp-tokens:
	preprocessing-token
	pp-tokens preprocessing-token
new-line:
	the new-line character
\end{code_files}




%%%%%%%%%%%%%%%%%%%%%%%%%%%%%%%%%%%%%%%%%%%%%%%%%%%%%%%%%%%%%%%%%%%%%%%%%%%%%%%%%%%%%%%%%%%%%%%%%%%%%%%%%%%%%%%%%%
\end{document} 
