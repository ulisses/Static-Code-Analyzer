\documentclass{beamer}

\mode<presentation>
{
   \usetheme{EEng}
 % \usetheme{Antibes}
  %\usecolortheme{lily}
  % \usetheme{Warsaw}
  \setbeamercovered{transparent}
  \setbeamercolor{background canvas}{bg=black!0}
}

%\usetheme{Antibes} % Beamer theme v 3.0
%\usecolortheme{lily}

\usepackage{enumerate}
\usepackage{array}
\usepackage{graphics}
\usepackage{ucs}
\usepackage[utf8x]{inputenc}
\usepackage[english]{babel}
\usepackage{amsmath, amsthm, amssymb}
\usepackage{amsmath}
\usepackage{amsfonts}
\usepackage{xcolor}
\usepackage{pgf}
\usepackage{hyperref}
\usepackage{url}
\usepackage{multicol}   % add-on
\usepackage{boxedminipage} 
\usepackage{indentfirst}   % add-on
\usepackage{float}
\usepackage{listings}
\usepackage{verbatim}
\usepackage{boxedminipage}
\usepackage{multimedia}

% % % inicio do listings e ref
\definecolor{darkblue}{rgb}{0,0,0.6}
\definecolor{gray_ulisses}{gray}{0.55}
\definecolor{castanho_ulisses}{rgb}{0.71,0.33,0.14}
\definecolor{preto_ulisses}{rgb}{0.41,0.20,0.04}
\definecolor{green_ulises}{rgb}{0.2,0.75,0}

\hypersetup{
    a4paper,
    pdftex,
    bookmarks,
    colorlinks,
    citecolor=darkblue,
    linkcolor=darkblue,
    urlcolor=darkblue,
    filecolor=darkblue
}

\lstdefinelanguage{perl_u} {
       basicstyle=\ttfamily\tiny,
       showstringspaces=false,
       breaklines=true,
       numbers=left,
       numberstyle=\tiny,
       numberblanklines=true,
       showspaces=false,
       showtabs=false
}
% % % fim do listings e ref

%\begin{figure}[htbp]
%\begin{center}
%\includegraphics[width=0.9\textwidth]{images/DI-UM.PNG}
%\end{center}
%\end{figure}

\title{Layar - Augmented Reality}
\author{José Pedro Silva \and
Pedro Faria \and
Ulisses Costa
}

\date{\today}
\institute{Engenharia de Linguagens\\
Processamento Estruturado de Documentos
}

\AtBeginSubsection[] {
  \begin{frame}<beamer>
    \frametitle{Index}
    \scriptsize{\tableofcontents[currentsection,currentsubsection]}
  \end{frame}
}

\AtBeginSection[] {
  \begin{frame}<beamer>
    \frametitle{Index}
    \scriptsize{\tableofcontents[currentsection]}
  \end{frame}
}
\begin{document}
\begin{frame}
   \titlepage
\end{frame}

\setbeamercovered{transparent=1}



\section{Layar}
\subsection{O que é o Layar?}

\begin{frame} \frametitle{O que é o Layar?}
\begin{itemize}
\item platafoma móvel que permite descobrir informação sobre o mundo à nossa volta
\pause \item usa realidade aumentada, para apresentar "camadas" de informação digital no campo  de visão do utilizador através do seu dispositivo movél
\pause \item é uma plataforma aberta que permite publicar, descobrir e procurar "camadas"
\pause \item só temos que criar as "camadas", o resto é fornecido pela aplicação \textit{Layar}
\pause \item \textit{Layar} é o vizualisador de realidade aumentada mais usado (1.4 milhões de utilizadores)
\pause \item está disponível para várias plataformas (Android, iPhone, Symbian, Bada)
\pause \item mais de 2000 "camadas" publicadas
\end{itemize}
\end{frame}

\subsection{Funcionalidades}
\begin{frame} \frametitle{Funcionalidades}
\begin{itemize}
\item suporte a  2D e a modelos 3D
\pause \item suporte para audio/vídeo
\pause \item acções disponíveis nas "camadas": ligar, e-mail, partilhar (Twitter, Facebook), aceder à Web Page  
\pause \item Suportado em 15 línguas incuindo o Português
\end{itemize}
\end{frame}

\subsection{Requisitos}
\begin{frame} \frametitle{Requisitos}
Dispostivo móvel com:
\begin{itemize}
\pause \item GPS - localização
\pause \item bússola - para onde estou a olhar
\pause \item câmara - captar a realidade
\pause \item acelerómetro - orientação do telefone
\pause \item conectividade à internet - aceder às "camadas" digitais
\pause \item giroscópio - para uma experiência mais fluida  (opcional)
\end{itemize}
\end{frame}


\section{Case-study}
\begin{frame} \frametitle{Case-study}
\begin{itemize}
\item Funcionário PT
\pause \item Utiliza o seu dispositivo móvel para encontrar e identificar caixas PT
\pause \item Ao apontar para uma caixa ou fio de telefone deverá receber informações sobre o mesmo (id da caixa, número e tipo das ligações,etc.)
\end{itemize}
\end{frame}


\section{Como utilizar?}
\begin{frame} \frametitle{Como utilizar?}
\begin{itemize}
\item criar uma "camada" no site do \textit{Layar}
\pause \item criar um serviço que devolve os POI's (Points of Interest) para a aplicação \textit{Layar}
		\begin{itemize}
			\pause \item web server público
			\pause \item base de dados
			\pause \item resposta é em JSON
		\end{itemize}
\pause \item A "camada" pode ser testada no site do \textit{Layar} antes de ser publicada
\end{itemize}
\end{frame}

\begin{frame} \frametitle{Base de dados}
\begin{block}{campos da tabela POI}
\textbf{id} - id único dentro da "camada"\\
\textbf{title} - titulo do painel de informações do POI\\
attribution,line1,line2,lin3,lin4 - atributos referentes aos restantes campos do painel com informações sobre o POI\\
\textbf{lat,lon} - latitude/longitude\\
\textbf{ImageURL} - link para a imagem a ser apresentada no painel de informações\\
\textbf{type} - define o tipo do POI (o tipo altera o tipo de icon)\\
\textbf{dimension} - \textbf{1}-\textgreater  icon, \textbf{2}-\textgreater  imagem 2D, \textbf{3}-\textgreater   objecto 3D\\
\textbf{alt} - (Optional) Real altitude of object in meters. If this is missing, the same altitude as the user is assumed.\\
\textbf{relativeAlt} - (Optional)Altitude in meters of the object with respect to the user's altitude.\\
\textbf{distance} - The distance of the POI to the current location\\

...
\end{block}
\end{frame}


\section*{Perguntas}
\begin{frame} \frametitle{Perguntas}
\begin{center}\huge{?}\end{center}
\end{frame}


\end{document}

